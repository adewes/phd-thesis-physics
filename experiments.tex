\chapter{Experiments}

%-Discuss all the experiments performed during the PhD thesis.

\section{Realizing a 2-Qubit Quantum Processor}

%-Motivate the performed experiments.

\subsection{Requirements}

%-List requirements for diVincenzo-style quantum computation:
%-Good 1 & 2 Qubit Gates
%-Qubits can be reset
%-Individual Single Shot Readout with High Fidelity

\subsection{Design \& Implementation}

%-Discuss the design & realization of our 2-qubit processor:
%  -Chip design
%  -Analytical model and parameter design.
%  -Measurement setup & RF chain

\section{Characterization of the Processor}

%-Show basic characteristics of the processor:
%  -Qubit transition energies vs. fluxes
%  -Qubit fast frequency controls
%  -2 Qubit interaction

\subsection{Readout}

%-Discuss the readout errors and crosstalk

\subsection{Single-Qubit Manipulation}

%-Discuss single qubit manipulation, gate fidelity and state tomography
%Data: 14/12/2010

\subsection{Two-Qubit Manipulation}

%-Discuss the realization of a 2 qubit gate:
%  -Principle
%  -Implementation & Pulse Sequency
%  -Characterization through Quantum Process Tomography:
%     -Principles: State tomography, Pauli set, process tomography
%     -Discuss alternative representations of the process information:
%        -Chi matrix, Choi matrix, S, log S, Kraus operator representation
%  		-Errors: Discuss simulations, error models and possible reasons for discrepancies

%-Discuss the generation of Bell states, the measurement of entanglement witnesses and the measured violation of the CHSH equation.


%To Do:
% -Look through the swapping data and see if there's evidence for wavefunction collapse in qubit 2 when qubit 1 is measured and the two qubits are only partially entangled. Compare the Leo's proposition.

\section{Running Grover's Search Algorithm}

%Motivate this experiment:
% -Benchmark for superconducting quantum computers
% -Speed-up for searching in an unsorted database

\subsection{Introduction \& Motivation}

%-Explain the Grover experiment...
%		-Theoretical interest
%		-First demonstration in NMR
%   -Potential speed-up
%   -Details of the algorithm:
%     -Elementary operations
%     -Pulse shapes, corrections, ...

\subsection{Experimental Implementation}

%-Show the implementation principle of the experiment.
%  -Break down the algorithm using the universal quantum gates that we've implemented

\subsection{Results}

%To Do:
%  -Create figures for all steps of the algorithm using Matplotlib
%  -Re-Analyze the data using Denis' Mathematica
%-Discuss the results and errors.

\subsection{Conclusions}

%-Conclusions regarding quantum speed-up and applicability of results to larger-scale quantum computing.
