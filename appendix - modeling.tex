\chapter{Modeling of Multi-Qubit Systems}

%-Discuss the motivation for analyzing the multi-qubit systems studied in this work.

\section{Analytical Approach}

\subsection{Multi-Qubit Hamiltonian}

%-Discuss the generation of an analytical model for a n-qubit system with a quantum bus.

\subsection{Energies and Eigenstates}

%-Show the eigen-energies and eigen-states of the Hamiltonian for interesting cases.

\section{Master Equation Approach} \label{section:three_level_simulation}

%-Discuss the master equation approach
%
\begin{equation}
\hat{H} = \left( \begin{array}{ccccccccc}
										e_0^I+e_0^{II} \\
										& e_{0}^I+e_1^{II} \\
										& & e_{0}^I+e_2^{II} \\
										& & & e_{1}^I+e_0^{II} \\
										& & & & e_{1}^I+e_1^{II} \\
										& & & & & e_{1}^I+e_2^{II} \\
										& & & & & & e_{2}^{I}+e_0^{II} \\
										& & & & & & & e_{2}^{I}+e_1^{II} \\
										& & & & & & & & e_{2}^{I}+e_2^{II} \\
									\end{array}
					\right)
\end{equation}
%
Without loss of generality we can assume $e_0^{I}=0$, $e_0^{II}=0$. We define $\omega_{01}^I=e_1^{I}$, $\omega_{01}^{II}=e_1^{II}$, $\Delta_{01} = \omega_{01}^{II}-\omega_{01}^I$ and $\alpha^I = e_2^I-2e_1^{I}$ and $\alpha^{II} = e_2^{II}-2e_1^{II}$. The interaction Hamiltonian is given as
%
\begin{equation}
\hat{H}_i = \left(
			\begin{array}{ccccccccc}
				0 & 0 & 0 & g & 0 & 0 & 0 & 0 & 0 \\
				0 & 0 & 0 & 0 & 0 & 0 & 0 & 0 & 0 \\
				0 & 0 & 0 & 0 & \sqrt{2}g & 0 & 0 & 0 & 0 \\
				0 & g & 0 & 0 & 0 & 0 & 0 & 0 & 0 \\
				0 & 0 & \sqrt{2}g & 0 & 0 & 0 & \sqrt{2}g & 0 & 0 \\
				0 & 0 & 0 & 0 & 0 & 0 & 0 & 2g & 0 \\
				0 & 0 & 0 & 0 & \sqrt{2}g & 0 & 0 & 0 & 0 \\
				0 & 0 & 0 & 0 & 0 & 2g & 0 & 0 & 0 \\
				0 & 0 & 0 & 0 & 0 & 0 & 0 & 0 & 0 \\
			\end{array}
		\right)
\end{equation}
%
Going to the interaction picture with $\hat{H}_0 = \hat{H}$ the interaction Hamiltonian $\hat{H}_i$ becomes
%
\begin{equation}
\hat{H}_i = \left(
			\begin{array}{lllllllll}
				0 & \hdots \\
				0 & 0 & \hdots \\
				0 & 0 & 0 & \hdots \\
				0 & ge^{-i\Delta t} & 0 & 0 & \hdots  \\
				0 & 0 & \sqrt{2}ge^{-i(\Delta-\alpha^{II}) t} & 0 & 0 & \hdots  \\
				0 & 0 & 0 & 0 & 0 & 0 & \hdots \\
				0 & 0 & 0 & 0 & \sqrt{2}ge^{-i(\Delta+\alpha^I) t} & 0 & 0 & \hdots \\
				0 & 0 & 0 & 0 & 0 & 2ge^{-i(\Delta+\alpha^{I}-\alpha^{II}) t} & 0 & 0 & \hdots \\
				0 & 0 & 0 & 0 & 0 & 0 & 0 & 0 & 0 \\
			\end{array}
		\right)
\end{equation}
%
\subsubsection{Relaxation and Dephasing}

We can model relaxation and dephasing of the three-level system using a set of 6 Lindblad operators. The relaxation operators are based on the three matrices
%
\begin{align}
\sigma^-_{01} & =  \left( \begin{array}{ccc} 0 & 1 & 0 \\ 0 & 0 & 0 \\ 0 & 0 & 0 \end{array} \right) & \sigma^-_{12} & =  \left( \begin{array}{ccc} 0 & 0 & 0 \\ 0 & 0 & 1 \\ 0 & 0 & 0 \end{array} \right) &
\sigma^-_{02} & =  \left( \begin{array}{ccc} 0 & 0 & 1 \\ 0 & 0 & 0 \\ 0 & 0 & 0 \end{array} \right)
\end{align}
%
and the dephasing operators on
%
\begin{align}
\sigma^z_{01} & =  \left( \begin{array}{ccc} 1 & 0 & 0 \\ 0 &-1 & 0 \\ 0 & 0 & 0 \end{array} \right) &
\sigma^z_{12} & =  \left( \begin{array}{ccc} 0 & 0 & 0 \\ 0 & 1 & 0 \\ 0 & 0 &-1 \end{array} \right) &
\sigma^z_{02} & =  \left( \begin{array}{ccc} 1 & 0 & 0 \\ 0 & 0 & 0 \\ 0 & 0 &-1 \end{array} \right)  
\end{align}
%
Using these matrices, we define a set of three relaxation operators $L^r_{01} = \sqrt{\Gamma^r_{10}}\sigma^-_{01}$, $L^r_{12} = \sqrt{\Gamma^r_{12}}\sigma^-_{12}$ and $L^r_{02}=\sqrt{\Gamma^r_{02}}\sigma^-_{02}$ as well as a set of dephasing operators $L^\phi_{01} = \sqrt{\Gamma^\phi_{01}/2}\sigma^z_{01}$, $L^\phi_{12} = \sqrt{\Gamma^\phi_{12}/2}\sigma^z_{12}$, $L^\phi_{02}=\sqrt{\Gamma^\phi_{02}/2}\sigma^z_{02}$.

\subsection{Direct Integration}

We can use the Hamiltonian and the Lindblad operators discussed above to directly integrate the master equation of the two-qubit system, which is given as
%
\begin{equation}
\frac{d\rho}{dt} = -\frac{i}{\hbar}[H,\rho]+\sum\limits_j\left[2L_j \; \rho \; L_j^\dagger -\{L_j^\dagger L_j,\rho\}\right] \label{eq:master_equation_lindblad_form}
\end{equation}
%
Here $-i/\hbar \;[H,\rho]$ corresponds to the unitary evolution of the state of the system and the part right to it to the non-unitary part, i.e. relaxation and dephasing. For numerical analysis it is usually convenient to rewrite this equation in the form
%
\begin{equation}
\frac{d\vec{\rho}}{dt} = {\cal L}(t)\vec{\rho}(t) \label{eq:master_equation_superoperator_form}
\end{equation}
%
where $\vec{\rho}$ is a column vector containg all elements of $\rho$ and $\cal L$ is the so-called ``superoperator'' which acts on the vectorized density matrix. For a density matrix with dimension $N\times N$, the superoperator has dimensions $N^2 \times N^2$, which makes the numerical solution of eq. (\ref{eq:master_equation_superoperator_form}) computationally expensive for large $N$. An alternative method which avoids this inefficiency is the so-called ``quantum monte carlo'' method which we will discuss in the next section.

\smallskip

From these matrices, the corresponding two-qubit operator can be obtained as $L^I_i = L_i\otimes {\cal I}$ and $L^{II}_i = L_i\otimes {\cal I}$.

%-Show the rsults of the direct ingegration approach.

\begin{SCfigure}[1][ht!]
	\centering
	\includegraphics[width=0.7\textwidth]{"./material/figures/appendix/modeling/master equation/swap_01_10_m1010"}
	\caption[...]{...}
	\label{fig:master_equation_simulation_swap_01_10_10}
\end{SCfigure}

\begin{figure}[ht!]
	\centering
	\includegraphics[width=\textwidth]{"./material/figures/appendix/modeling/master equation/swap_11_02_20/swap_11_02_20"}
	\caption[...]{...}
	\label{fig:master_equation_simulation_swap_11_02_20}
\end{figure}

\subsection{Monte Carlo Simulation}

%-Show the results of the Monte Carlo method.

\subsection{Speeding Up Simulations}

%-Discuss acceleration of simulation by GPGPU techniques.

\begin{lstlisting}[language=python]
from helpnet import *

def foo(x,y):
	print "Hello, world. This is quite a long line which should be wrapped I guess. The sum x + y = %g" % (x,y)
\end{lstlisting}
