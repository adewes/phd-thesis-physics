\chapter{Conclusions \& Outlook} \label{chapter:conclusions}

In this thesis work, we implemented a two-qubit processor using superconducting Transmon qubits that are capacitively coupled to each other. Each of the qubit possesses its own single-shot QND readout realized using a Josephson bifurcation amplifier. Using the qubit-qubit interaction, we implemented the universal $\sqrt{i\mathrm{SWAP}}$ quantum gate and used it to create entangled two-qubit states. In addition, we implemented the Grover quantum search algorithm, demonstrating probabilistic quantum speed-up beyond the classical boundary for a four-element search problem.

\smallskip

Despite our successful implementation of a simple ``toy'' quantum processor, the approach that we have chosen for its realization is fundamentally flawed and can thus not be used to build larger quantum processors: The fixed, always-ON type of qubit-qubit coupling employed in our processor makes it impossible to realiable switch on and off the coupling between individual qubits when increasing their number. Furthermore, the qubit coherence times that we achieved in our experiments are not sufficient to run large-scale algorithms.

\smallskip

To overcome this problems, it would be necessary to increase the coherence times of superconducting qubits by a large factor and to devise a quantum computing architecture that solves the coupling problem. In chapter \ref{chapter:scalable_architecture} of this work, we propose a qubit architecture that can potentially provide a tunable coupling scheme and would therefore be scalable beyond a small number of qubits. This architecture is compatible with any kind of Transmon qubits and could potentially be combined e.g. with recently developed 3D Transmons, which themselves exhibit very large coherence times. Together, long coherence times and tunable qubit-qubit coupling would probably allow the realization of a superconducting qubit processor with a fews 10s of qubits that could compete with current state-of-the-art ion qubit quantum processors. 

\smallskip

Overall, considering the recent progress that has been made on various issues related to superconducting qubits, the realization of a superconducting quantum computer seems no longer an impossible task, although a daunting one.