%% LyX 1.6.9 created this file.  For more info, see http://www.lyx.org/.
%% Do not edit unless you really know what you are doing.
\documentclass[english,aps]{revtex4}
\usepackage[T1]{fontenc}
\usepackage[latin9]{inputenc}
\usepackage{textcomp}
\usepackage{amstext}
\usepackage{graphicx}

\makeatletter

%%%%%%%%%%%%%%%%%%%%%%%%%%%%%% LyX specific LaTeX commands.
%% A simple dot to overcome graphicx limitations
\newcommand{\lyxdot}{.}


%%%%%%%%%%%%%%%%%%%%%%%%%%%%%% Textclass specific LaTeX commands.
\@ifundefined{textcolor}{}
{%
 \definecolor{BLACK}{gray}{0}
 \definecolor{WHITE}{gray}{1}
 \definecolor{RED}{rgb}{1,0,0}
 \definecolor{GREEN}{rgb}{0,1,0}
 \definecolor{BLUE}{rgb}{0,0,1}
 \definecolor{CYAN}{cmyk}{1,0,0,0}
 \definecolor{MAGENTA}{cmyk}{0,1,0,0}
 \definecolor{YELLOW}{cmyk}{0,0,1,0}
 }

\makeatother

\usepackage{babel}

\begin{document}

\title{Supplementary Information for\\
Characterization of a two-transmon processor with individual single-shot
qubit readout}


\author{A. Dewes$^{1}$, F. R. Ong$^{1}$, V. Schmitt$^{1}$, R. Lauro$^{1}$,
N. Boulant$^{2}$, P. Bertet$^{1}$, D. Vion$^{1}$, and D. Esteve$^{1}$}


\affiliation{$^{1}$Quantronics group, Service de Physique de l'�tat Condens�
(CNRS URA 2464), IRAMIS, DSM, CEA-Saclay, 91191 Gif-sur-Yvette, France }


\affiliation{$^{2}$ I2BM, Neurospin, LRMN, 91191CEA-Saclay, 91191 Gif-sur-Yvette,
France }


\date{\today}

\maketitle
S1. Sample preparation\\


The sample is fabricated on a silicon chip oxidized over 50 nm. A
150 nm thick niobium layer is first deposited by magnetron sputtering
and then dry-etched in a $SF_{6}$ plasma to pattern the readout resonators,
the current lines for frequency tuning, and their ports. Finally,
the transmon qubit, the coupling capacitance and the Josephson junctions
of the resonators are fabricated by double-angle evaporation of aluminum
through a shadow mask patterned by e-beam lithography. The first layer
of aluminum is oxidized in a $Ar-O_{2}$ mixture to form the oxide
barrier of the junctions. The chip is glued with wax on a printed
circuit board (PCB) and wire bonded to it. The PCB is then screwed
in a copper box anchored to the cold plate of a dilution refrigerator.\\


S2. Sample parameters\\


The sample is first characterized by spectroscopy (see Fig.$\,$1.b
of main text). The incident power used is high enough to observe the
resonator frequency $\nu_{\mathrm{R}}$, the qubit line $\nu_{01}$,
and the two-photon transition at frequency $\nu_{02}/2$ between the
ground and second excited states of each transmon (data not shown).
A fit of the transmon model to the data yields the sample parameters
$E_{\mathrm{J}}^{\mathrm{I}}/h=36.2\,\mathrm{GHz}$, $E_{\mathrm{C}}^{\mathrm{I}}/h=0.98\,\mathrm{GHz}$,
$d_{I}=0.2$, $E_{\mathrm{J}}^{\mathrm{II}}/h=43.1\,\mathrm{GHz}$,
$E_{\mathrm{C}}^{\mathrm{II}}/h=0.87\,\mathrm{GHz}$, $d_{\mathrm{II}}=0.35$,
$\nu_{\mathrm{R}}^{\mathrm{I}}=6.84\,\mathrm{GHz}$, and $\nu_{\mathrm{R}}^{\mathrm{II}}=6.70\,\mathrm{GHz}$.
The qubit-readout anticrossing at $\nu=\nu_{\mathrm{R}}$ yields the
qubit-readout couplings $g_{0}^{\mathrm{I}}\simeq g_{0}^{\mathrm{II}}\simeq50\,\mathrm{MHz}$.
Independent measurements of the resonator dynamics (data not shown)
yield quality factors $Q_{\mathrm{I}}=Q_{\mathrm{II}}=730$ and Kerr
non linearities {[}13,\cite{FlorianKerr}{]} $K_{\mathrm{I}}/\nu_{\mathrm{R}}^{\mathrm{I}}\simeq K_{\mathrm{II}}/\mathrm{\nu}_{\mathrm{R}}^{II}\simeq-2.3\pm0.5\times10^{-5}$.\\


S3. Experimental setup
\begin{itemize}
\item Qubit microwave pulses: The qubit drive pulses are generated by two
phase-locked microwave generators whose continuous wave outputs are
fed to a pair of I/Q-mixers. The two IF inputs of each of these mixers
are provided by a 4-Channel$1\,\mathrm{GS/s}$ arbitrary waveform
generator (AWG Tektronix AWG5014). Single-sideband mixing in the frequency
range of 50-300 MHz is used to generate multi-tone drive pulses and
to obtain a high ON/OFF ratio ($>\,50\,\mathrm{dB}$) of the signal
at the output of the mixers. Phase and amplitude errors of the mixers
are corrected by measuring the signals at the output and applying
sideband and carrier frequency dependent corrections in amplitude
and offset to the IF input channels. 
\item Flux Pulses: The flux control pulses are generated by a second AWG
and sent to the chip through a transmission line, equipped with 40
dB of attenuation distributed over different temperature stages and
a pair of 1 GHz absorptive low-pass filters at $4\,\mathrm{K}$. The
input signal of each flux line is fed back to room temperature through
an identical transmission line and measured to compensate the non-ideal
frequency response of the line.
\item Readout Pulses: The pulses for the Josephson bifurcation amplifier
(JBA) readouts are generated by mixing the continuous signals of a
pair of microwave generators with IF pulses provided by a $1\,\mathrm{GS/s}$
arbitrary function generator. Each readout pulse consists of a measurement
part with a rise time of $30\,\mathrm{ns}$ and a hold time of 100
ns, followed by a $2\,\mu s$ long latching part at 90 \% of the pulse
height. 
\item Drive and Measurement Lines: The drive and readout microwave signals
of each qubit are combined and sent to the sample through a pair of
transmission lines that are attenuated by 70 dB over different temperature
stages and filtered at $4\mathrm{\, K}$ and $300\,\mathrm{mK}$.
A microwave circulator at $20\,\mathrm{mK}$ separates the input signals
going to the chip from the reflected signals coming from the chip.
The latter are amplified by $36\,\mathrm{dB}$ at $4\,\mathrm{K}$
by two cryogenic HEMT amplifiers (CIT Cryo 1) with noise temperature
$5\,\mathrm{K}$. The reflected readout pulses get further amplified
at room temperature and demodulated with the continuous signals of
the readout microwave sources. The IQ quadratures of the demodulated
signals are sampled at $1\,\mathrm{GS/s}$ by a 4-channel Data Acquisition
system (Acqiris DC282). 
\end{itemize}
S4. Readout characterization\\


Errors in our readout scheme are discussed in detail in {[}13{]} for
a single qubit. First, incorrect mapping $\left|0\right\rangle \rightarrow1$
or$\left|1\right\rangle \rightarrow0$ of the projected state of the
qubit to the dynamical state of the resonator can occur, due to the
stochastic nature of the switching between the two dynamical states.
As shown in Fig. S4.1, the probability $p$ to obtain the outcome
1 varies continuously from 0 to 1 over a certain range of drive power
$P_{\mathrm{d}}$ applied to the readout. When the shift in power
between the two $p_{\left|0\right\rangle ,\left|1\right\rangle }(P_{\mathrm{d}})$
curves is not much larger than this range, the two curves overlap
and errors are significant even at the optimal drive power where the
difference in $p$ is maximum. Second, even in the case of non overlapping
$p_{\left|0\right\rangle ,\left|1\right\rangle }(P_{\mathrm{d}})$
curves, the qubit initially projected in state$\left|1\right\rangle $
can relax down to $\left|0\right\rangle $ before the end of the measurement,
yielding an outcome 0 instead of 1. The probability of these two types
of errors vary in opposite directions as a function of the frequency
detuning $\Delta=\nu_{\mathrm{R}}-\nu>0$ between the resonator and
the qubit, so that a compromise has to be found for $\Delta$. Besides,
the contrast $c=Max\left(p_{\left|1\right\rangle }-p_{\left|0\right\rangle }\right)$
can be increased {[}12{]} by shelving state $\left|1\right\rangle $
into state $\left|2\right\rangle $ with a microwave $\pi$ pulse
at frequency $\nu_{12}$ just before the readout resonator pulse.
The smallest errors $e_{0}^{\mathrm{I,II}}$ and $e_{1}^{\mathrm{I,II}}$
when reading $\left|0\right\rangle $ and $\left|1\right\rangle $
are found for $\Delta_{\mathrm{I}}=440\,\mathrm{MHz}$ and $\Delta_{\mathrm{II}}=575\,\mathrm{MHz}$
and are shown by arrows in the top panels of Fig. S3.2: $e_{0}^{\mathrm{I}}=5\%$
and $e_{1}^{\mathrm{I}}=13\%$ (contrast $c_{\mathrm{I}}=1-e_{0}^{\mathrm{I}}-e_{1}^{\mathrm{I}}=82\%$),
and $e_{0}^{\mathrm{II}}=5.5\%$ and $e_{1}^{\mathrm{II}}=12\%$ ($c_{\mathrm{II}}=82\%$).
When using the $\left|1\right\rangle \rightarrow\left|2\right\rangle $
shelving before readout, $e_{0}^{\mathrm{I}}=2.5\%$ and $e_{2}^{\mathrm{I}}=9.5\%$
(contrast $c_{\mathrm{I}}==1-e_{0}^{\mathrm{I}}-e_{2}^{\mathrm{I}}=88\%$),
and $e_{0}^{\mathrm{II}}=3\%$ and $e_{2}^{\mathrm{II}}=8\%$ ($c_{\mathrm{II}}=89\%$).
These best results are very close to those obtained in {[}12{]}, but
are unfortunately not relevant to this work.

%
\begin{figure}[tbph]
\includegraphics[width=12cm]{\string"U:/Commun/PROJET Cavit�s/articles/ISWAP/figures/FigureS4-1\string".eps}

\caption{Readout imperfections and their correction. (a) Switching probabilities
of the readouts as a function of their driving power, with the qubit
prepared in state$\left|0\right\rangle $ (blue), $\left|1\right\rangle $
( red), or $\left|2\right\rangle $ (brown), at the optimal readout
points. The arrows and dashed segments indicate the readout errors
and contrast, at the power where the later is maximum. (b) Same as
(a) but at readout points $R^{\mathrm{I,II}}$ used in this work.
(c-d) Single readout matrices $\mathcal{C}_{\mathrm{I,II}}$ and pure
readout crosstalk matrix $\mathcal{C}_{\mathrm{CT}}$ characterizing
the simultaneous readout of the two qubits. (e-g) bare readout outcomes
$uv$, outcomes corrected from the independent readout errors only,
and$\left|uv\right\rangle $ population calculated with the full correction
including crosstalk for the swapping experiment of Fig. 2.}


\label{fig S4.1}%
\end{figure}


Indeed, when the two qubits are measured simultaneously, one has also
to take into account a possible readout crosstalk, i.e. an influence
of the projected state of each qubit on the outcome of the readout
of the other qubit. We do observe such an effect and have to minimize
it by increasing $\Delta_{\mathrm{I,II}}$ up to $\sim1\,\mathrm{GHz}$
with respect to previous optimal values and by not using the shelving
technique. An immediate consequence shown in Fig. 3.2 (b) is a reduction
of the $c_{\mathrm{I,II}}$ contrasts. The errors when reading $\left|0\right\rangle $
and $\left|1\right\rangle $ are now $e_{0}^{\mathrm{I}}=19\,\%$
and $e_{1}^{\mathrm{I}}=7\,\%$ (contrast $c_{\mathrm{I}}=74\%$)
and $e_{0}^{\mathrm{II}}=19\,\%$ and $e_{1}^{\mathrm{II}}=12\,\%$
(contrast $c_{\mathrm{II}}=69\%$). Then to characterize the errors
due to crosstalk, we measure the $4\times4$ readout matrix $\mathcal{R}$
linking the probabilities $p_{\mathrm{uv}}$ of the four possible
$uv$ outcomes to the population of the four $\left|uv\right\rangle $
states. As shown in Fig. S3.2(c-d), we then rewrite $\mathcal{R}=\mathcal{C}_{\mathrm{CT}}.\left(\mathcal{C}_{\mathrm{I}}\otimes\mathcal{C}_{\mathrm{II}}\right)$
as the product of a $4\times4$ pure crosstalk matrix $\mathcal{C}_{\mathrm{CT}}$
with the tensorial product of the two $2\times2$ single qubit readout
matrices \[
\mathcal{C}_{I,II}=\left(\begin{array}{cc}
1-e_{0}^{\mathrm{I,II}} & e_{1}^{\mathrm{I,II}}\\
e_{0}^{\mathrm{I,II}} & 1-e_{1}^{\mathrm{I,II}}\end{array}\right).\]


We also illustrate on the figure the impact of the readout errors
on our swapping experiment by comparing the bare readout outcomes
$uv$, the outcomes corrected from the independent readout errors
only, and the$\left|uv\right\rangle $ population calculated with
the full correction including crosstalk.

We now explain briefly the cause of the readout crosstalk in our processor.
Unlike what was observed for other qubit readout schemes using switching
detectors {[}5{]}, the crosstalk we observe is not directly due to
an electromagnetic perturbation induced by the switching of one detector
that would help or prevent the switching of the other one. Indeed,
when both qubits frequencies $\nu_{\mathrm{I,II}}$ are moved far
below $\nu_{\mathrm{R}}^{\mathrm{I,II}}$, the readout crosstalk disappears:
the switching of a detector has no measurable effect on the switching
of the other one. The crosstalk is actually due to the rather strong
ac-Stark shift $\sim2\left(n_{\mathrm{H}}-n_{\mathrm{L}}\right)g_{0}^{2}/(R-\nu_{\mathrm{R}})\sim500\,\mathrm{MHz}$
of the qubit frequency when a readout resonator switches from its
low to high amplitude dynamical state with $n_{\mathrm{L}}\sim10$
and $n_{\mathrm{H}}\sim10^{2}$ photons, respectively. The small residual
effective coupling between the qubits at readout can then slightly
shift the frequency of the other resonatator, yielding a change of
its switching probability by a few percent. Note that coupling the
two qubits by a resonator rather than by a fixed capacitor would solve
this problem.\\


S5. Removing errors on tomographic pulses before calculating the gate
process map\\


%
\begin{figure}[tbph]
\includegraphics[width=17cm]{\string"U:/Commun/PROJET Cavit�s/articles/ISWAP/figures/FigureS5.1\string".eps}

\caption{Fitting of the pulse errors at state preparation and tomography. Measured
(red) and fitted (blue - see text) Pauli sets $\left\langle P_{\mathrm{k}}^{\mathrm{e}}\right\rangle $
for the sixteen targeted input states $\{\left|0\right\rangle ,\left|1\right\rangle ,\left|0\right\rangle +\left|1\right\rangle ,\left|0\right\rangle +i\left|1\right\rangle \}^{\otimes2}$.
The $\{II,IX,IY,IZ,XI,...\}$ operators indicated in abscisse are
the targeted operators and not those actually measured (due to tomographic
errors).}


\label{fig S5.1}%
\end{figure}
Tomographic errors are removed from the process map of our $\sqrt{iSWAP}$
gate using the following method. The measured Pauli sets corresponding
to the sixteen input states are first fitted by a model including
errors both in the preparation of the state (index $prep$) and in
the tomographic pulses (index $tomo$). The errors included are angular
errors $\varepsilon_{\mathrm{I,II}}^{\mathrm{prep}}$ on the nominal
$\pi$ rotations around $X_{\mathrm{I,II}}$, $\eta_{\mathrm{I,II}}^{\mathrm{prep,tomo}}$and
$\delta_{\mathrm{I,II}}^{\mathrm{prep,tomo}}$ on the nominal $\pi/2$
rotations around $X_{\mathrm{I,II}}$ and $Y_{\mathrm{I,II}}$, a
possible departure $\xi_{\mathrm{I,II}}$ from orthogonality of $\left(\overrightarrow{X_{\mathrm{I}}},\overrightarrow{Y_{\mathrm{I}}}\right)$
and $\left(\overrightarrow{X_{\mathrm{II}}},\overrightarrow{Y_{\mathrm{II}}}\right)$,
and a possible rotation $\mu_{\mathrm{I,II}}$ of the tomographic
$XY$ frame with respect to the preparation one. The rotation operators
used for preparing the states and doing their tomography are thus
given by

\[
\begin{array}{c}
X_{\mathrm{I,II}}^{\mathrm{prep}}(\pi)=e^{-\mathrm{i}\left(\pi+\varepsilon_{\mathrm{I,II}}^{\mathrm{prep}}\right)\sigma_{\mathrm{x}}^{\mathrm{I,II}}/2},\\
X_{\mathrm{I,II}}^{\mathrm{prep}}(-\pi/2)=e^{+\mathrm{i}\left(\pi/2+\eta_{\mathrm{I,II}}^{\mathrm{prep}}\right)\sigma_{\mathrm{x}}^{\mathrm{I,II}}/2},\\
Y_{\mathrm{I,II}}^{\mathrm{prep}}(\pi/2)=e^{-\mathrm{i}\left(\pi/2+\delta_{\mathrm{I,II}}^{\mathrm{prep}}\right)\left[\mathrm{cos}\left(\xi_{\mathrm{I,II}}\right)\sigma_{\mathrm{y}}^{\mathrm{I,II}}\mathrm{-sin}\left(\xi_{\mathrm{I,II}}\right)\sigma_{\mathrm{x}}^{\mathrm{I,II}}\right]/2},\\
X_{\mathrm{I,II}}^{\mathrm{tomo}}(\pi/2)=e^{-\mathrm{i}\left(\pi/2+\eta_{\mathrm{I,II}}^{\mathrm{tomo}}\right)\left[\mathrm{\mathrm{sin}\left(\mu_{I,II}\right)\sigma_{x}^{I,II}+cos}\left(\mu_{\mathrm{I,II}}\right)\sigma_{\mathrm{y}}^{\mathrm{I,II}}\right]/2},\\
Y_{\mathrm{I,II}}^{\mathrm{tomo}}(-\pi/2)=e^{+\mathrm{i}\left(\pi/2+\delta_{\mathrm{I,II}}^{\mathrm{tomo}}\right)\left[\mathrm{cos}\left(\mu_{\mathrm{I,II}}+\xi_{\mathrm{I,II}}\right)\sigma_{\mathrm{y}}^{\mathrm{I,II}}\mathrm{-sin}\left(\mu_{\mathrm{I,II}}+\xi_{\mathrm{I,II}}\right)\sigma_{x}^{\mathrm{I,II}}\right]/2}.\end{array}\]
The sixteen input states are then $\left\{ \rho_{\mathrm{in}}^{\mathrm{e}}=U\left|0\right\rangle \left\langle 0\right|U^{\dagger}\right\} $
with $\left\{ U\right\} =\{I_{\mathrm{I}},X_{\mathrm{I}}^{\mathrm{prep}}(\pi),Y_{\mathrm{I}}^{\mathrm{prep}}(\pi/2),X_{\mathrm{I}}^{\mathrm{prep}}(-\pi/2)\}\otimes\{I_{\mathrm{II}},X_{\mathrm{II}}^{\mathrm{prep}}(\pi),Y_{\mathrm{II}}^{\mathrm{prep}}(\pi/2),X_{\mathrm{II}}^{\mathrm{prep}}(-\pi/2)\}$,
and each input state yields a Pauli set $\left\{ \left\langle P_{\mathrm{k}}^{\mathrm{e}}\right\rangle =Tr\left(\rho_{\mathrm{in}}^{\mathrm{e}}P_{\mathrm{k}}^{\mathrm{e}}\right)\right\} $
with $\left\{ P_{\mathrm{k}}^{\mathrm{e}}\right\} =\{I_{\mathrm{I}},X_{\mathrm{I}}^{\mathrm{e}},Y_{\mathrm{I}}^{\mathrm{e}},Z_{\mathrm{I}}\}\otimes\{I_{\mathrm{II}},X_{\mathrm{II}}^{\mathrm{e}},Y_{\mathrm{II}}^{\mathrm{e}},Z_{\mathrm{II}}\}$,
$X^{\mathrm{e}}=Y^{\mathrm{tomo}}(-\pi/2)^{\dagger}\sigma_{z}Y^{\mathrm{tomo}}(-\pi/2)$,
and $Y^{\mathrm{e}}=X^{\mathrm{tomo}}(\pi/2)^{\dagger}\sigma_{\mathrm{z}}X^{\mathrm{tomo}}(\pi/2)$.
Figure S5.1 shows the best fit of the modelled $\left\{ \left\langle P_{k}^{e}\right\rangle \right\} $
set to the measured input Pauli sets, yielding $\varepsilon_{\mathrm{I}}^{\mathrm{prep}}=-1\text{\textdegree}$,
$\varepsilon_{\mathrm{II}}^{\mathrm{prep}}=-3\text{\textdegree}$,
$\eta_{\mathrm{I}}^{\mathrm{prep}}=3\text{\textdegree}$, $\mathrm{\eta}_{\mathrm{II}}^{\mathrm{prep}}=4\text{\textdegree}$,
$\delta_{\mathrm{I}}^{\mathrm{prep}}=-6\text{\textdegree}$, $\delta_{\mathrm{II}}^{\mathrm{prep}}=-3\text{\textdegree}$,
$\eta_{\mathrm{I}}^{\mathrm{tomo}}=-6\text{\textdegree}$, $\eta_{\mathrm{II}}^{\mathrm{tomo}}=-4\text{\textdegree}$,
$\lambda_{\mathrm{I}}^{t\mathrm{omo}}=12\text{\textdegree}$, $\lambda_{\mathrm{II}}^{\mathrm{tomo}}=5\text{\textdegree}$,
$\xi_{\mathrm{I}}=1\text{\textdegree}$, $\xi_{\mathrm{II}}=-2\text{\textdegree}$,
and $\mu_{\mathrm{I}}=\mu_{\mathrm{II}}=-11\text{\textdegree}$.

Knowing the tomographic errors and thus $\left\{ \left\langle P_{\mathrm{k}}^{\mathrm{e}}\right\rangle \right\} $,
we then invert the linear relation $\left\{ \left\langle P_{\mathrm{k}}^{\mathrm{e}}\right\rangle =Tr\left(\rho P_{\mathrm{k}}^{\mathrm{e}}\right)\right\} $
to find the $16\times16$ matrix $B$ that links the vector $\overrightarrow{\left\langle P_{\mathrm{k}}^{\mathrm{e}}\right\rangle }$
to the columnized density matrix $\overrightarrow{\rho}$, i.e. $\overrightarrow{\rho}=B.\overrightarrow{\left\langle P_{\mathrm{k}}^{\mathrm{e}}\right\rangle }$.
The matrix $B$ is finally applied to the measured sixteen input and
sixteen output Pauli sets to find the sixteen $(\rho_{\mathrm{in},},\rho_{\mathrm{out}})_{\mathrm{k}}$
couples to be used for calculating the gate map.
\begin{thebibliography}{1}
\bibitem{FlorianKerr} F. R. Ong \textit{et al.}, Phys. Rev. Lett.
106, 167002 (2011).
\end{thebibliography}

\end{document}
