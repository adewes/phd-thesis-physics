%% LyX 1.6.9 created this file.  For more info, see http://www.lyx.org/.
%% Do not edit unless you really know what you are doing.
\documentclass[twocolumn,english,aps,twocolumn]{revtex4}
\usepackage[T1]{fontenc}
\usepackage[latin9]{inputenc}
\usepackage{color}
\usepackage{textcomp}
\usepackage{amsthm}
\usepackage{amsmath}
\usepackage{graphicx}
\usepackage{amssymb}
\usepackage{esint}

\makeatletter

%%%%%%%%%%%%%%%%%%%%%%%%%%%%%% LyX specific LaTeX commands.
\newcommand{\lyxmathsym}[1]{\ifmmode\begingroup\def\b@ld{bold}
  \text{\ifx\math@version\b@ld\bfseries\fi#1}\endgroup\else#1\fi}


%%%%%%%%%%%%%%%%%%%%%%%%%%%%%% Textclass specific LaTeX commands.
\@ifundefined{textcolor}{}
{%
 \definecolor{BLACK}{gray}{0}
 \definecolor{WHITE}{gray}{1}
 \definecolor{RED}{rgb}{1,0,0}
 \definecolor{GREEN}{rgb}{0,1,0}
 \definecolor{BLUE}{rgb}{0,0,1}
 \definecolor{CYAN}{cmyk}{1,0,0,0}
 \definecolor{MAGENTA}{cmyk}{0,1,0,0}
 \definecolor{YELLOW}{cmyk}{0,0,1,0}
 }

\makeatother

\usepackage{babel}

\begin{document}

\title{Characterization of a two-transmon processor with individual single-shot
qubit readout }


\author{A. Dewes$^{1}$, F. R. Ong$^{1}$, V. Schmitt$^{1}$, R. Lauro$^{1}$,
N. Boulant$^{2}$, P. Bertet$^{1}$, D. Vion$^{1}$, and D. Esteve$^{1}$}


\affiliation{$^{1}$Quantronics group, Service de Physique de l'�tat Condens�
(CNRS URA 2464), IRAMIS, DSM, CEA-Saclay, 91191 Gif-sur-Yvette, France }


\affiliation{$^{2}$LRMN, Neurospin, I2BM, DSV , 91191CEA-Saclay, 91191 Gif-sur-Yvette,
France }


\date{\today}
\begin{abstract}
We report the characterization of a two-qubit processor implemented
with two capacitively coupled tunable superconducting qubits of the
transmon type, each qubit having its own non-destructive single-shot
readout. The fixed capacitive coupling yields the $\sqrt{iSWAP}$
two-qubit gate for a suitable interaction time. We reconstruct by
state tomography the coherent dynamics of the two-bit register as
a function of the interaction time, observe a violation of the Bell
inequality by 22 standard deviations after correcting readout errors,
and measure by quantum process tomography a gate fidelity of 90\%.
\end{abstract}
\maketitle
\textcolor{black}{Quantum information processing is one of the most
appealing ideas for exploiting the resources of quantum physics and
performing tasks beyond the reach of classical machines \cite{Nielsen Chuang}.
Ideally, a quantum processor consists of an ensemble of highly coherent
two-level systems, the qubits, that can be efficiently reset, that
can follow any unitary evolution needed by an algorithm using a universal
set of single and two qubit gates, and that can be readout projectively.
In the domain of electrical quantum circuits \cite{QubitsSupra},
important progress \cite{DiCarlo2qb,Yamamoto,Bialczak,IBM Steffen,DiCarlo3qb}
has been achieved recently with the operation of elementary quantum
processors based on different superconducting qubits. Those based
on transmon qubits \cite{DiCarlo2qb,DiCarlo3qb,Transmon Koch,TransmonSchreier}
are well protected against decoherence but embed all the qubits in
a single resonator used both for coupling them and for joint readout.
Consequently, individual readout of the qubits is not possible and
the results of a calculation, as the Grover search algorithm demonstrated
on two qubits \cite{DiCarlo2qb}, cannot be obtained by running the
algorithm only once. Furthermore, the overhead for getting a result
from such a processor without single-shot readout but with a larger
number of qubits overcomes the speed-up gain expected for any useful
algorithm. The situation is different for processors based on phase
qubits \cite{Yamamoto,Bialczak,RezQ}, where the qubits are more sensitive
to decoherence but can be read individually with high fidelity, although
destructively. This significant departure from the wished scheme can
be circumvented, when needed, since a destructive readout can be transformed
into a non-destructive one at the cost of adding one ancilla qubit
and one extra two-qubit gate for each qubit to be read projectively.
Moreover, energy release during a destructive readout can result in
a sizeable cross-talk between the readout outcomes, which can also
be solved at the expense of a more complex architecture \cite{Ansmann Bell,RezQ}. }

%
\begin{figure}[t]
\includegraphics[width=8cm]{\string"U:/Commun/PROJET Cavit�s/articles/ISWAP/figures/Figure1\string".eps}\caption{(a): circuit schematics \textcolor{black}{of the experiment with qubits
in green and readout circuits in grayed blue}. (b) Left panel: Spectroscopy
of the sample showing the resonator frequencies $\nu_{\mathrm{R}}^{\mathrm{\mathrm{I},II}}$
(horizontal lines), and the measured (disks, triangles)\textcolor{black}{{}
and fitted (lines) }qubit frequencies $\nu_{\mathrm{I,II}}$ as a
function of their flux bias $\phi_{\mathrm{I,II}}$ when the other
qubit is far detuned. Right panel: Spectroscopic anticrossing of the
two qubits revealed by the 2D plot of $p_{\mathrm{01}}+p_{\mathrm{10}}$
as a function of the probe frequency and of $\phi_{\mathrm{I}}$,
at $\nu_{\mathrm{II}}=5.124\,\mathrm{GHz}$. (c) Typical pulse sequence
including $X$ or $Y$ rotations, a $\sqrt{iSWAP}{}^{\alpha}$gate,
$Z$ rotations, and tomographic and readout pulses. Microwave pulses
$a(t)$ for qubit (green) and for readout (blue) are drawn on top
of the $\nu_{\mathrm{I,II}}(\phi)$ dc pulses (red lines).}


\label{fig1}%
\end{figure}
\textcolor{black}{In this work, we operate a new architecture that
comes closer to the ideal quantum processor design than the above-mentioned
ones. Our circuit is based on frequency tunable transmons that are
capacitively coupled. Although the coupling is fixed, the interaction
is effective only when the qubits are on resonance, which yields the
$\sqrt{iSWAP}$ universal gate for an adequate coupling duration.
Each qubit is equipped with its own non-destructive single-shot readout
\cite{JBA Siddiqi,JBA Mallet} and the two qubits can be read with
low cross-talk. In order to characterize the circuit operation, we
reconstruct the time evolution of the two-qubit register density matrix
during the resonant and coherent exchange of a single quantum of excitation
between the qubits by quantum state tomography. Then, we prepare a
Bell state with concurrence 0.85, measure the CHSH entanglement witness,
and find a violation of the corresponding Bell inequality by 22 standard
deviations. We then characterize the $\sqrt{iSWAP}$ universal gate
operation by determining its}\textcolor{magenta}{{} }\textcolor{black}{process
map with quantum process}\textcolor{red}{{} }\textcolor{black}{tomography
\cite{Nielsen Chuang}. We find a gate fidelity of 90\% due to qubit
decoherence and systematic unitary errors. }

The circuit implemented is schematized in Fig.\ref{fig1}a: the \textcolor{black}{coupled
}qubits with their respective \textcolor{black}{control and readout
sub-circuits are} fabricated on a Si chip (see supplementary information
S1). The chip is cooled down to $20\,\mathrm{mK}$ in a dilution refrigerator
and connected to room temperature sources and measurement devices
by attenuated and filtered control lines and by two measurement lines
equipped with cryogenic amplifiers. Each transmon $j=I,\, II$ is
a capacitively shunted SQUID characterized by its Coulomb energy $E_{\mathrm{C}}^{\mathrm{j}}$
for a Cooper pair, the asymmetry $d_{\mathrm{j}}$ between its two
Josephson junctions, and its total effective Josephson energy $E_{\mathrm{J}}^{\mathrm{j}}(\phi_{j})=E_{\mathrm{J}}^{\mathrm{j}}\left|\mathrm{cos}(x_{\mathrm{j}})\right|\sqrt{1+d_{\mathrm{j}}^{2}\mathrm{tan^{2}}(x_{\mathrm{j}})}$,
with $x_{\mathrm{j}}=\pi\phi_{\mathrm{j}}/\phi_{0}$, $\phi_{0}$
the flux quantum, and $\phi_{\mathrm{j}}$ the magnetic flux through
the SQUIDs induced by two local current lines with a $0.5\,\mathrm{GHz}$
bandwidth. The transition frequencies $\nu_{\mathrm{j}}\simeq\sqrt{2E_{\mathrm{C}}^{\mathrm{j}}E_{\mathrm{J}}^{\mathrm{j}}}/h$
between the two lowest energy states $\left|0\right\rangle _{j}$
and $\left|1\right\rangle _{j}$ can thus be tuned by $\phi_{\mathrm{j}}$.
The qubits are coupled by a capacitor with nominal value $C_{c}\simeq0.13\,\mathrm{fF}$
and form a register with Hamiltonian $H=h\left(-\nu_{\mathrm{I}}\sigma_{\mathrm{z}}^{\mathrm{I}}-\nu_{\mathrm{II}}\sigma_{\mathrm{z}}^{\mathrm{II}}+{\color{black}{\color{red}{\color{black}2}}}g\sigma_{\mathrm{y}}^{\mathrm{I}}\sigma_{\mathrm{y}}^{\mathrm{II}}\right)/2$.
Here $h$ is the Planck constant, $\sigma_{\mathrm{x,y,z}}$ are the
Pauli operators, ${\color{red}{\color{black}2}}g=\sqrt{E_{\mathrm{C}}^{\mathrm{I}}E_{\mathrm{C}}^{\mathrm{II}}\nu_{\mathrm{I}}\nu_{\mathrm{II}}}/E_{\mathrm{Cc}}\ll\nu_{\mathrm{I,II}}$
is the coupling frequency, and $E_{\mathrm{Cc}}$ the Coulomb energy
of a Cooper pair on the coupling capacitor. The two-qubit gate is
defined in the uncoupled basis $\{\left|uv\right\rangle \}\equiv\{\left|0\right\rangle _{\mathrm{I}},\left|1\right\rangle _{\mathrm{I}}\}\otimes\{\left|0\right\rangle _{\mathrm{II}},\left|1\right\rangle _{\mathrm{II}}\}$,
at a working point $M_{\mathrm{I,II}}$ where the qubits are sufficiently
detuned ($\nu_{\mathrm{II}}-\nu_{\mathrm{I}}\gg{\color{red}{\color{black}2}}g$)
to be negligibly coupled. Bringing them on resonance at a frequency
$\nu$ in a time much shorter than $1/2g$ but much longer than 1/$\nu$,
and keeping them on resonance during a time $\Delta t$, one implements
an operation $\Theta_{\mathrm{I}}.\Theta_{\mathrm{II}}.\sqrt{iSWAP}^{\,(8g\Delta t)}$,
which is the product of the\[
\sqrt{iSWAP}=\left(\begin{array}{cccc}
1 & 0 & 0 & 0\\
0 & 1/\sqrt{2} & -i/\sqrt{2} & 0\\
0 & -i/\sqrt{2} & 1/\sqrt{2} & 0\\
0 & 0 & 0 & 1\end{array}\right)\]
gate to an adjustable power and of two single qubit phase gates $\Theta_{\mathrm{j}}=\mathrm{exp\left(i\theta_{j}\sigma_{z}^{j}{\color{red}{\color{black}/2}}\right)}$
accounting for the dynamical phases $\theta_{\mathrm{j}}=\int{\color{red}{\color{black}2}}\pi(\nu-\nu_{\mathrm{j}})dt$
accumulated during the coupling. The exact $\sqrt{iSWAP}$ gate can
thus be obtained by choosing $\Delta t={\color{red}{\color{black}1}}/8g$
and by applying a compensation rotation $\Theta_{\mathrm{j}}^{-1}$
to each qubit afterward.

For readout, each qubit is capacitively coupled to its own \textcolor{black}{$\lambda/2$
}coplanar waveguide resonator \textcolor{black}{with frequency $\nu_{\mathrm{R}}^{\mathrm{j}}$
and quality factor $Q_{\mathrm{j}}$. This resonator is made non linear
with a Josephson junction and }is operated as a Josephson bifurcation
amplifier, as explained in detail in \cite{JBA Mallet}. The homodyne
measurement \textcolor{black}{(see Fig.\ref{fig1}a) }of two microwave
pulses simultaneously applied to and reflected from the resonators
yields a two-bit outcome $uv$ that maps with a high fidelity the
state $\left|uv\right\rangle $ on which the register is projected;
the probabilities $p_{\mathrm{uv}}$ of the four possible outcomes
are determined by repeating the same experimental sequence a few $10^{4}$
times. Single qubit rotations $u\left(\theta\right)$ by an angle
$\theta$ around an axis $\overrightarrow{u}$ of the XY plane of
the Bloch sphere are obtained by applying Gaussian microwave pulses
through the readout resonators, with frequencies $\nu_{\mathrm{j}}$,
phases $\varphi_{\mathrm{j}}=(\overrightarrow{X},\overrightarrow{u})$,
and calibrated area $A_{\mathrm{j}}\varpropto\theta$. Rotations around
Z are obtained by changing temporarily $\nu_{\mathrm{I,II}}$ with
dc pulses on the current lines.

The sample is first characterized by spectroscopy (see Fig.\ref{fig1}b)
and a fit of the transmon model to the data yields the sample parameters
(see S2). The working points where the qubits are manipulated $(M^{\mathrm{I,II}}$),
resonantly coupled (C), and read out $(R^{\mathrm{I,II}}$) are chosen
to yield sufficiently long relaxation times $\sim0.5\,\text{\textmu s}$
during gates, negligible residual coupling during single qubit rotations
and readout, and best possible fidelities at readout. Figure \ref{fig1}b
shows these points as well as the spectroscopic anticrossing of the
two qubits at point $C$, where ${\color{red}{\color{black}2}}g=8.3\;\mathrm{MHz}$
in agreement with the design value of $C_{\mathrm{c}}$. Then, readout
errors are characterized at $R^{\mathrm{I,II}}$ (see Fig. S3.1):
In a first approximation, the errors are independent for the two readouts
and are of about 10\% and 20\% when reading $\left|0\right\rangle $
and $\left|1\right\rangle $ respectively. This limited fidelity results
for a large part from energy relaxation of the qubits at readout.
In addition we observe a small readout cross talk, i.e. a variation
of up to 2\% in the probability of an outcome of readout $j$ depending
on the state of the other qubit. All these effects are calibrated
by measuring the four $p_{\mathrm{uv}}$ probabilities for each of
the four $\left|uv\right\rangle $ states, which allows us to calculate
a $4\times4$ readout matrix $\mathcal{R}$ linking the $p_{\mathrm{uv}}$'s
to the $\left|uv\right\rangle $ populations.

%
\begin{figure}[h]
\includegraphics[width=8cm]{\string"U:/Commun/PROJET Cavit�s/articles/ISWAP/figures/Figure2\string".eps}\caption{Coherent swapping of a single excitation between the qubits. (a) Experimental
(solid lines) and fitted (dashed lines) occupation probabilities of
the four computational states $\left|00\right\rangle ..\left|11\right\rangle $
as a function of the coupling duration. No $Z$ or tomographic pulses
are applied here. (b,c) State tomography of the initial state (left)
and of the state produced by the $\sqrt{iSWAP}$ gate (right). (b)
Ideal (empty bars) and experimental (color filling) expected values
of the 15 Pauli operators $XI,..,ZZ$. (c) Corresponding ideal (color
filled black circles with pointer) and experimental (red circle and
pointer) density matrices, as well as fidelity $F$ and concurence
$C$. Each matrix element is represented by a circle with an area
proportional to its modulus (diameter = cell size for unit modulus)
and a phase pointer giving its argument.}


\label{fig2}%
\end{figure}


Repeating the pulse sequence shown in Fig.\ref{fig1}c at $M^{\mathrm{I}}=5.247\,\mathrm{GHz}$,
$M^{\mathrm{II}}=C=5.125\,\mathrm{GHz}$, $R^{\mathrm{I}}=5.80\,\mathrm{GHz}$,
$R^{\mathrm{II}}=5.75\,\mathrm{GHz}$, and applying the readout corrections
$\mathcal{R}$, we observe the coherent exchange of a single excitation
initially stored in qubit $I$. We show in Fig.\ref{fig2} the time
evolution of the measured $\left|uv\right\rangle $ populations, in
fair agreement with a prediction obtained by integration of a simple
time independent Liouville master equation of the system, involving
the independently measured relaxation times $T_{1}^{\mathrm{I}}=436\,\mathrm{ns}$
and $T_{1}^{\mathrm{II}}=520\,\mathrm{ns}$, and two independent effective
pure dephasing times $T_{\mathrm{\varphi}}^{\mathrm{I}}=T_{\mathrm{\varphi}}^{\mathrm{II}}=2.0\,\mu s$
as fitting parameters. Tomographic reconstruction of the register
density matrix $\rho$ is obtained by measuring the expectation values
of the 15 two-qubit Pauli operators $\left\{ P_{\mathrm{k}}\right\} =\left\{ XI,..,ZZ\right\} $,
the $X_{\mathrm{j}}$ and $Y_{\mathrm{j}}$ measurements being obtained
using tomographic pulses $\overrightarrow{Y_{\mathrm{j}}}\left(-90\text{\textdegree}\right)$
or $\overrightarrow{X_{\mathrm{j}}}\left(90\text{\textdegree}\right)$
just before readout. The $\rho$ matrix is calculated from the Pauli
set by global minimization of the Hilbert-Schmidt distance between
the possibly non-physical $\rho$ and all physical (i.e. positive-semidefinite)
$\rho's$. This can be done at regular interval of the coupling time
to produce a movie of $\rho\left(\Delta t\right)$ (see supplementary
on line material) showing the swapping of the $\left|10\right\rangle $
and $\left|01\right\rangle $ populations at frequency ${\color{red}{\color{black}2}}g$,
the corresponding oscillation of the coherences, as well as the relaxation
towards $\left|00\right\rangle $. Figure \ref{fig2} shows $\left\{ \left\langle P_{\mathrm{k}}\right\rangle \right\} $
and $\rho$ only at $\Delta t$=$0\,\mathrm{ns}$ and after a $\sqrt{iSWAP}$
obtained at $\Delta t$=$31\,\mathrm{ns}$ with $\Theta_{\mathrm{j}}^{-1}$rotations
of $\theta_{\mathrm{I}}\simeq-65\text{\textdegree}$ and $\theta_{\mathrm{II}}\simeq+60\text{\textdegree}$.
The fidelity $F=\left\langle \psi_{\mathrm{id}}\right|\rho\left|\psi_{\mathrm{id}}\right\rangle $
of $\rho$ with the ideal density matrices $\left|\psi_{\mathrm{id}}\right\rangle \left\langle \psi_{\mathrm{id}}\right|$
are 95\% and 91\%, respectively, and are limited by errors on the
preparation pulse, statistical noise, and relaxation.

%
\begin{figure}
\includegraphics[width=8cm]{\string"U:/Commun/PROJET Cavit�s/articles/ISWAP/figures/Figure3\string".eps}

\caption{Test of the CHSH-Bell inequality on a $\left|10\right\rangle +e^{\mathrm{i\psi}}\left|01\right\rangle $
state by measuring the qubits along $X^{\mathrm{I}}$or $Y^{\mathrm{I}}$
and $\mathrm{X_{\mathrm{\varphi}}^{II}}$ or $Y_{\mathrm{\varphi}}^{\mathrm{II}}$
(see top-left inset)\textcolor{black}{,}\textcolor{red}{{} }\textcolor{black}{respectively}.
Blue (resp. red) error bars are the experimental CHSH entanglement
witness determined from the raw (resp. readout errors corrected) measurements
as a function of the angle $\varphi$ between the measuring basis,
whereas solid line is a fit using $\psi$ as the only fitting parameter.
Height of error bars is $\pm$ one standard deviation $\sigma(N)$
(see bottom-right inset), with $N$ the number of sequences per point.
Note that averaging beyond $N=10^{6}$ does not improve the violation
because of a slow drift of $\varphi$.}


\label{fig3}%
\end{figure}


To quantify in a different way our ability to entangle the two qubits,
we prepare a Bell state $\left|10\right\rangle +e^{\mathrm{i\psi}}\left|01\right\rangle $
\textcolor{black}{(with }$\psi=\theta_{\mathrm{II}}-\theta_{\mathrm{I}}$)\textcolor{black}{{}
using the pulse sequence of Fig.\ref{fig1}c with $\Delta t=31\,\mathrm{ns}$
and no $\Theta_{j}^{-1}$rotations}, and measure the CHSH entanglement
witness $\left\langle XX_{\mathrm{\varphi}}\right\rangle +\left\langle XY_{\mathrm{\varphi}}\right\rangle +\left\langle YY_{\mathrm{\varphi}}\right\rangle -\left\langle YX_{\mathrm{\varphi}}\right\rangle $
as a function of the angle $\varphi$ between the orthogonal measurement
bases of qubit $I$ and $II$. Figure \ref{fig3} compares the results
obtained with and without correcting the readout errors, with what
is theoretically expected from the decoherence parameters indicated
previously: unlike in \cite{Ansmann Bell} and because of a readout
contrast limited to $70-75\,\%$, the witness does not exceed the
classical bound of 2 without correcting the readout errors. After
correction, it reaches 2.43, in good agreement with the theoretical
prediction (see also \cite{Chow Bell}), and exceeds the classical
bound by up to 22 standard deviations when averaged over 10$^{6}$
sequences. 

%
\begin{figure}[h]
\includegraphics[width=8cm]{\string"U:/Commun/PROJET Cavit�s/articles/ISWAP/figures/Figure4\string".eps}

\caption{Map of the implemented $\sqrt{iSWAP}$ gate yielding a fidelity of
90\%. (a) Superposition of the ideal (empty thick bars) and experimental
(color filled bars) lower part of the Hermitian matrix $\chi$ (elements
below 1\% not shown). Each complex matrix element is represented by
a bar with height proportional to its modulus and a red phase pointer
at the top of the bar (as well as a filling color for experiment)
giving its argument (top left inset). Expected peaks are marked by
a star. (b) Lower part of the $\widetilde{\chi}$ error matrix (red
circles - see text), with the same convention as in Fig.\ref{fig2},
but with circles magnified for readability (a one-cell diameter represents
a 8 \% modulus). Main visible contributions (continuous circles) are
explained in text.}


\label{fig4}%
\end{figure}


In a last experiment, we characterize the imperfections of our $\sqrt{iSWAP}$
gate by quantum process tomography \cite{Nielsen Chuang}. We build
a completely positive map $\rho_{\mathrm{out}}=\mathcal{E}(\rho_{\mathrm{in}})=\sum_{\mathrm{m,n}}\chi_{\mathrm{mn}}P_{\mathrm{m}}^{'}\rho_{\mathrm{in}}P_{\mathrm{n}}^{'\dagger}$
characterized by a $16\times16$ matrix $\chi$ expressed here in
the modified Pauli operator basis $\left\{ P_{\mathrm{k}}^{'}\right\} =\{I,X,Y^{'}=iY,Z\}^{\otimes2}$,
for which all matrices are real. For that purpose, we apply the gate
(\textcolor{black}{using pulse sequences similar to that of \ref{fig1}c,
with $\Delta t=31\,\mathrm{ns}$ and $\Theta_{\mathrm{j}}^{-1}$rotations})
to the sixteen input states $\{\left|0\right\rangle ,\left|1\right\rangle ,\left|0\right\rangle +\left|1\right\rangle ,\left|0\right\rangle +i\left|1\right\rangle \}^{\otimes2}$
and characterize both the input and output states by quantum state
tomography. By operating as described previously, we would obtain
apparent input and output density matrices including errors made in
the state tomography itself, which we don't want to include in the
gate map. Instead, we fit the 16 experimental input Pauli sets by
a model including amplitude and phase errors for the $X$ and $Y$
preparation and tomographic pulses (see S4), in order to determine
which operator set $\left\{ P_{\mathrm{k}}^{\mathrm{e}}\right\} $
is actually measured. The input and output matrices $\rho_{\mathrm{in,out}}$
corrected from the tomographic errors only are calculated by inverting
the linear relation $\left\{ \left\langle P_{\mathrm{k}}^{\mathrm{e}}\right\rangle =Tr\left(\rho P_{\mathrm{k}}^{\mathrm{e}}\right)\right\} $
and by applying it to the experimental Pauli sets. We then calculate
from the $\left\{ \rho_{\mathrm{in,out}}\right\} $ set an Hermitian
$\chi$ matrix that is not necessarily physical due to statistical
errors, and which we render physical by taking the nearest Hermitian
positive matrix. This final $\chi$ matrix is shown in Fig. \ref{fig4}
and compared to the ideal one, $\chi_{\mathrm{id}}$, which yields
a gate fidelity $F_{\mathrm{g}}=Tr\left(\chi.\chi_{\mathrm{id}}\right)=0.9$
\cite{Fidelities}. To better understand the imperfections, we also
show the map $\widetilde{\chi}$ of the actual process preceded by
the inverse ideal process \cite{Korotkov}. The first diagonal element
of $\widetilde{\chi}$ is equal to $F$ by construction. Then, main
visible errors arise from unitary operations and reduce fidelity by
1-2\% (a fit yields a too long coupling time inducing a 95\textdegree{}
swap instead of 90\textdegree{} and $\Theta_{1,2}$ rotations too
small by $3.5\lyxmathsym{\textdegree}$ and $7\lyxmathsym{\textdegree}$
respectively). On the other hand the known relaxation and dephasing
times reduces fidelity by 8\% but is barely visible in $\widetilde{\chi}$
due to a spread over many matrix elements with modulus of the order
of or below the $1-2\%$ noise level. 

As a conclusion, we have demonstrated a high fidelity $\sqrt{iSWAP}$
gate in a two josephson qubit circuit with individual non-destructive
single-shot readouts, observed a violation of the CHSH-Bell inequality,
and followed the register's dynamics by tomography. Although quantum
coherence and readout fidelity are still limited in this circuit,
they are sufficient to test in the near future simple quantum algorithms
and get their result in a single run, which would demonstrate the
concept of quantum speed-up.

We gratefully acknowledge discussions with J. Martinis and his coworkers,
with M. Devoret, D. DiVicenzo, A. Korotkov, P. Milman, and within
the Quantronics group, technical support from P. Orfila, P. Senat,
and J.C. Tack, as well as financial support from the European research
contracts MIDAS and SOLID, from ANR Masquelspec and C'Nano and from
the German Ministry of Education and Research.
\begin{thebibliography}{16}
\bibitem{Nielsen Chuang} M. A. Nielsen and I. L. Chuang, Quantum
Computation and Quantum Information (Cambridge University Press, Cambridge,
UK, 2000). 

\bibitem{QubitsSupra} J. Clarke and F. Wilhelm, Nature \textbf{453},
1031 (2008).

\bibitem{DiCarlo2qb} L. DiCarlo\emph{ et al.}, Nature \textbf{460},
240 (2008).

\bibitem{DiCarlo3qb} L. DiCarlo\emph{ et al.}, Nature \textbf{467},
574 (2010).

\bibitem{Yamamoto}T. Yamamoto \textit{et al}., Phys. Rev. B \textbf{82},
184515 (2010).

\bibitem{Bialczak}R. C. Bialczak\textit{ et al}., Nature Physics
\textbf{6}, 409-413 (2010).

\bibitem{IBM Steffen}J. M. Chow\emph{ et al.}, Phys. Rev. Lett. 107,
080502 (2011).

\bibitem{Transmon Koch}J. Koch\emph{ et al.}, Phys. Rev. A \textbf{76},
042319 (2007).

\bibitem{TransmonSchreier} J.A. Schreier \emph{et al.}, Phys. Rev.
B \textbf{77}, 180502(R) (2008).

\bibitem{Ansmann Bell}M. Ansmann\emph{ et al.}, Nature \textbf{461},
504 (2009).

\bibitem{RezQ}M. Mariantoni \emph{et al.}, Science, DOI: 10.1126/science.1208517
(2011).

\bibitem{JBA Siddiqi}I. Siddiqi \emph{et al.}, Phys. Rev. Lett. \textbf{93},
207002 (2004).

\bibitem{JBA Mallet}F. Mallet\emph{ et al.}, Nature Physics \textbf{5},
791 (2009).

\bibitem{Chow Bell}J. M. Chow\emph{ et al.}, Phys. Rev. A \textbf{81},
062325 (2010).

\bibitem{Fidelities}Note that $F_{g}$ is also equal to Shumacher's
fidelity $Tr\left[S_{id}^{\dagger}S\right]/Tr\left[S_{id}^{\dagger}S_{id}\right]$
with $S$ (resp. $S_{id}$) the super operator of the actual (resp.
ideal unitary) process, and that fidelities $F$ for the 16 outputs
states range between 80\% and 99.5\%

\bibitem{Korotkov}A. G. Kofman and A. N. Korotkov, Phys. Rev. A,
vol. \textbf{81}, no. 4, 042103 (2009), and private communication.
\end{thebibliography}

\end{document}
