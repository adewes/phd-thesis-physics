\chapter{Theoretical Foundations} \label{chapter:theory}

In this chapter we provide the reader with the minimum conceptual background and theoretical building blocks necessary  to understand this thesis work, which will be then presented in the following chapters. We begin our discussion with a general overview of classical and quantum information processing with a Turing machine, followed by an introduction to superconducting quantum circuits. We summarize the method to quantize electrical circuits and apply it to Cooper pair box devices, and in particular to the Transmon that will be used to implement the qubit register of our quantum processor. We then present coplanar waveguide resonators, and introduce circuit quantum electrodynamics on the example of a transmon coupled to a such a resonator. Finally, we consider the case of a nonlinear resonator used as a Josephson bifurcation amplifier since we use such a readout device in our processor architecture.

\section{Classical \& Quantum Information Processing}

\begin{SCfigure}
	\includegraphics[width=8cm]{"./material/figures/introduction/bloch_sphere"}
	\caption{The Bloch sphere representation of a qubit state $\ket{\psi} = \cos{\frac{\theta}{2}}\ket{0}+e^{i\phi}\sin{\frac{\theta}{2}}\ket{1}$. The state $\ket{\psi}$ is fully characterized by specifying its ``latitude'' and ``azimuth'' angles $\theta$ and $\phi$. Pure quantum states will always lie on the surface of the Bloch sphere, whereas mixed quantum states can also lie anywhere inside the sphere.}
	\label{fig:BlochSphere}
\end{SCfigure}

By definition, computing designates the activity of using computer hardware and software to process information, or {\it data}. Classical information processing can be divided in so-called {\it analog and digital information processing}, the former being based on continuously changeable physical quantities whereas the latter is based on incrementally changeable quantities. The fundamental unit of digital information processing is the so-called {\it bit}, which represents a Boolean (true/false) information. The discipline of theoretical computer science has been created to investigate the fundamental limits and properties of classical information processing. One of the main foundational theorems of theoretical computer science is the so-called {\it Church-Turing thesis} which provides a universal computing model by saying (basically) that everything which is computable can be efficiently computed using a {\it Turing machine}. Such a Turing machine, in turn, is a simple theoretical device which is able to run programs that operate on a discrete set of data using a well-specified set of operations. The Turing machine is universal in the sense that any other classical computing device can be efficiently emulated using a Turing machine with the appropriate program and data.

\smallskip

In the early 1980, Richard Feynman discovered that a classical Turing machine as described above would be unable to efficiently simulate a quantum-mechanical system \citep{feynman_simulating_1982}. He introduced the concept of a {\it quantum Turing machine} that would be able to simulate quantum-mechanical systems in an efficient manner. A few years later, David Deutsch took up Feynman's idea and developed an information processing framework based on quantum mechanics \citep{deutsch_quantum_1985}, coining the terms {\it quantum computing} and {\it quantum information processing}. He showed that by making use of different properties of quantum mechanics, namely, the superposition principle and entanglement, one could solve certain mathematical problems faster than possible with any classical computer \citep{deutsch_quantum_1985}. The work by Deutsch created a large interest in the physics community and led to a huge theoretical and experimental effort aimed at realizing an operational quantum computer and developing quantum algorithms for relevant real-world problems.

\section{Principles of "Conventional" Quantum Computing}

The first scheme imagined for quantum information processing is directly inspired from the classical digital Turing machine: information is stored in a set of quantum two level systems, the {\it quantum bits} or {\it qubits}, forming a quantum register, which is manipulated by sequentially applying unitary (and possibly non-unitary) operators to subsets of qubits in the register, typically one or two. As for a classical Turing machine, any arbitrary operation of this quantum processor can be decomposed as a sequence of gate operations chosen from a surprisingly small set of gates, said universal. The power of such a machine comes from the gates operating on superposed states, which provides an intrinsic parallelism in the processing.
\smallskip
This {\it quantum gate} approach is the method relevant to the present thesis work, and we ignore other approaches introduced more recently such as {\it one-way quantum computing} \citep{raussendorf_one-way_2001}, {\it adiabatic quantum computing} \citep{farhi_quantum_2000} or {\it topological quantum computing} \citep{kitaev_fault-tolerant_2003}. In this section we introduce very briefly quantum bits and quantum gates, as well as some examples of quantum algorithms that are relevant to this work.


\subsection{Quantum Bits and registers}

As in classical computing, one can define in quantum computing a fundamental unit of information: the qubit. Such a qubit is a quantum-mechanical two-level system that can be put in any superposition
%
\begin{equation}
\ket{\psi} = \cos{\frac{\theta}{2}}\ket{0}+e^{i\phi}\sin{\frac{\theta}{2}}\ket{1}
\end{equation}
%
of its two level states $\ket{0}$ and $\ket{1}$.
As can be seen, any state can be described by a pair of real numbers $\theta$ and $\phi$ that characterize the amplitude of each of the two basis states and the phase between them. A useful and intuitive representation of such a single-qubit state is the so-called {\it Bloch sphere representation}, shown in fig. \ref{fig:BlochSphere}. In this representation, any pure state $\ket{\psi}$ is located on a unit sphere. The north and south poles of this sphere correspond by convention to the qubit states $\ket{0}$ and $\ket{1}$ (or vice versa). All states lying on the sphere between those two correspond to superposition states, which are characterized by their ``latitude'' and ``azimuth'' angles $\theta$ and $\phi$. 

\smallskip

\smallskip
If the qubit state is not pure but is a mixed state given by the complete set of probabilities ${p_{i}}$ to find it in one of the pure states  ${\ket{\psi_{i}}}$, it is characterized by the density matrix $\rho_{mixed} = \sum\limits_i p_i \rho_i$ with $\rho_i=\ket{\psi_{i}}\bra{\psi_{i}}$ the density matrix of the pure state $\ket{\psi_i}$. These $\rho$'s are Hermitian $2\times 2$ matrices with non-negative eigenvalues and can be written as
%
\begin{equation}
\rho = \left( \begin{array}{cc} \rho_{00} & \rho_{01} \\ \rho_{01}^* & \rho_{11} \end{array} \right),
\end{equation}
%
in the ${\ket{0},\ket{1}}$ basis, with $\rho_{00}$ and $\rho_{11}$ being real numbers and $\rho_{01}$ a complex number. For any state, the matrix has a unity trace $\mathrm{Tr}\{\rho\}=1$, for pure states we have $\rho=\rho^2$ in addition. The expectation values $\langle A \rangle$ of any operator $A$ acting on the density matrix $\rho$ is given as $\mathrm{Tr}(\rho A)$.

\smallskip

A mixed single-qubit state $\rho$ can also be represented on the Bloch sphere. For this, we decompose $\rho = c_i\sigma_I+c_x\sigma_x+c_y\sigma_y+c_z\sigma_z$, where $c_{i,x,y,z}$ are complex coefficients and the matrices $\sigma_I,\sigma_x,\sigma_y,\sigma_z$ are
%
\begin{align}
  \sigma_I  =  \left( \begin{array}{cc} 1 & 0 \\ 0 & 1 \end{array} \right)
 & &  \sigma_x  =  \left( \begin{array}{cc} 0 & 1 \\ 1 & 0 \end{array} \right)
  & & \sigma_y  =  \left( \begin{array}{cc} 0 & -i \\ i  &  0\end{array} \right)
  & & \sigma_z  =  \left( \begin{array}{cc} 1 & 0 \\ 0 & -1 \end{array} \right).
\label{eq:pauli_operators}
\end{align}
% 
We can then plot the vector $(c_x,c_y,c_z)$ on the Bloch sphere, which will lie inside or on the surface of the sphere. The length of this vector decreases from 1 for a pure state down to 0 for a completely mixed state.

\smallskip

Quantum states with more than one qubit cannot be described anymore using Bloch spheres but are easily described with density matrices of dimension $2^n\otimes 2^n$ in the computational basis $\ket{q_1\hdots q_n}=\ket{q_1}\otimes\ket{q_{2}}\hdots\otimes\ket{q_n}$.For example, a two-qubit Bell state of the form $\ket{\psi_+}=(\ket{01}+\ket{10})/\sqrt{2}$ can be written as
%
\begin{equation}
\rho_{\psi_+} = \frac{1}{2}\left( \begin{array}{cccc} 0 & 0 & 0 & 0 \\ 0 & 1 & 1 & 0 \\ 0 & 1 & 1 & 0 \\ 0 & 0 & 0 & 0 \end{array} \right)_{(\ket{00},\ket{01},\ket{10},\ket{11})}
\end{equation}
%
whereas a completely mixed state of 50 \% $\ket{01}$ and 50 \% $\ket{10}$ is represented by the matrix
%
\begin{equation}
\rho_{\psi_+} = \frac{1}{2}\left( \begin{array}{cccc} 0 & 0 & 0 & 0 \\ 0 & 1 & 0 & 0 \\ 0 & 0 & 1 & 0 \\ 0 & 0 & 0 & 0 \end{array} \right)
\end{equation}
%

\subsection{Quantum Gates}

Analogously to classical information processing one defines {\it quantum gates} that act on individual or multiple qubits and allow us to process information. Such a quantum gate can be described as a unitary operator acting on a part of the Hilbert space representing the qubit register. Theoretically there is an infinite number of possible quantum gates, however in order to describe all possible quantum operations that can be performed on a qubit register of arbitrary length it is sufficient to defined a so-called {\it universal set of quantum gates}. Such a set contains a small number of quantum gates that can, by concatenation, produce any arbitrary unitary quantum operator, as shown by the so-called {\it Solovay-Kitaev theorem} \citep{nielsen_quantum_2000,dawson_solovay-kitaev_2005}. Such a universal gate set that will be especially relevant to this work consists of the three single-qubit rotation matrices
%
\begin{eqnarray}
   R_x(\theta)  & = & e^{-i\sigma_x\frac{\phi_x}{2}} \\ 
   R_y(\theta)  & = & e^{-i\sigma_y\frac{\phi_y}{2}} \\ 
   R_z(\theta)  & = & e^{-i\sigma_z\frac{\phi_z}{2}} 
\label{eq:universal_single_qubit_gates}
\end{eqnarray}
%
together with the so-called $\sqrt{i\mathrm{SWAP}}$ two-qubit operator, which has the representation
%
\begin{equation}
\sqrt{i\mathrm{SWAP}} = \left( \begin{array}{cccc} 1 & 0 & 0 & 0 \\ 0 & 1/\sqrt{2} & i/\sqrt{2} & 0 \\ 0 & i/\sqrt{2} & 1/\sqrt{2} & 0 \\ 0 & 0 & 0 & 1  \end{array}  \right)_{(\ket{00},\ket{01},\ket{10},\ket{11})}
\end{equation}
%
This universal set it not minimal, since in principle two single-qubit gates with a fixed rotation angle (e.g. $R_x(\pi/4)$ and $R_y(\pi/4)$) together with a universal two-qubit gate would be sufficient to form a universal set of gates \citep{dawson_solovay-kitaev_2005}. However, it is often advantageous if one can use single-qubit rotations with arbitrary rotation angles around all three axes of the Bloch sphere since it can significantly reduce the number of gates required to implement a given unitary operation.

\subsection{Quantum Algorithms}

The interest in quantum computing is mainly due to the fact that certain problems can be solved faster on a quantum computer than on a classical computer. By faster we mean here that the order $\cal{O}$ of the run time of the algorithm increases faster on a classical computer than on a quantum computer as a function of the problem size, i.e. the number of bits needed to encode the problem. Up to this day it has not been demonstrated that a quantum computer can perform all tasks faster than a classical computer. However, a small number of real-world problems have been found that can be solved exponentially to polynomially faster on a quantum computer. Here we cite only the two most ``famous'' ones:

\begin{enumerate}
\item \textbf{The Shor Factorization Algorithm} Developed by Peter Shor in 1994 \citep{shor_algorithms_1994,shor_polynomial-time_1995}. This algorithm can factorize a binary number of length $N$ into its prime factors in $\begin{mathcal}O\end{mathcal}(\log^3{N})$ steps, therefore exponentially outperforming any known classical factorization algorithm. There is large interest in this algorithm since products of large prime numbers are routinely used in asymmetric cryptography.
\item \textbf{The Grover Search Algorithm}: Discovered by Lov Grover in 1996 \citep{grover_fast_1996}, this search algorithm can find a single well-defined state in an unsorted database of size $N$ in $\begin{mathcal}O\end{mathcal}(\sqrt{N})$ steps, being hence quadratically faster than a classical search algorithm.
\end{enumerate}

\subsection{Quantum Simulation}

Another domain of interest for quantum computers is the so called {\it quantum simulation} \citep{lloyd_universal_1996}. Here the goal is to simulate the behavior of an arbitrary quantum system using a quantum computer by either engineering the quantum computer in direct analogy with the system being modeled (so called {\it analog quantum simulation}) or by numerically simulating the Hamiltonian of the quantum system on a general-purpose quantum computer (so-called {\it digital quantum simulation}). Since no classical computer can simulate a quantum system efficiently, there is a large interest in quantum simulation, especially in the fields of biology \& chemistry \citep{barreiro_open-system_2011}, quantum field theory \citep{gerritsma_quantum_2010,freedman_simulation_2002} and many-body physics \citep{simon_quantum_2011}.

\subsection{Realization of a Quantum Computer}

To realize a working quantum computer, it is necessary to implement highly coherent qubits that can be manipulated, read out and coupled with high fidelity. So far, no fully working quantum computer has been experimentally demonstrated. However, larger progress towards its realization has been achieved in the last decade. Promising approaches for the realization of a quantum computer include --among others-- ions trapped in magnetic and electric fields \citep{monroe_demonstration_1995,cirac_quantum_1995}, nuclear magnetic resonance of organic molecules \citep{jones_nmr_2001,vandersypen_experimental_2001}, cold atomic gases \citep{briegel_quantum_2000}, photonic circuits \citep{knill_scheme_2001}, semiconductor circuits \citep{loss_quantum_1998} and, last but not least, superconducting circuits. Since this work treats only superconducting qubits of the Transmon type, we focus our attention on them in the following sections. We explain how we can realize a reliable qubit using superconducting structures and how we can implement circuits to manipulate, couple and read out the qubit state.

\section{Superconducting Quantum Circuits}

In this section we discuss several types of superconducting circuit elements that are most relevant to this work. First, we introduce the reader to the Josephson junction, which is the device we use to realize superconducting qubits and amplifiers. Then, we present a general method for the quantization of arbitrary electrical circuits that we use afterwards to perform canonical quantization of our circuits. We use this method to derive the Hamiltonian of the Cooper pair box and treat the Transmon qubit as a special case. Afterwards, we discuss the properties of transmission lines and transmission line resonators that we use extensively for implementing readout and coupling elements in our qubit design. Then we give a short overview of the field of circuit quantum electrodynamics and finally introduce the reader to the Josephson and cavity bifurcation amplifiers that we use for our qubit readout.

\subsection{The Josephson junction}

The core element used to construct quantum circuits is the so-called {\it Josephson junction}, being equivalent in significance to the transistor in classical circuits. A Josephson junction is based on the so-called Josephson effect \citep{josephson_possible_1962}, which states that between two superconductors connected through a {\it weak link}, a supercurrent
%
\begin{equation}
I = I_c\sin{\varphi}
\end{equation}
%
will flow, depending on the difference $\varphi = \varphi_2-\varphi_1$ between the gauge-invariant superconducting phases $\varphi_1$ and $\varphi_2$ at each side of the link. $I_c$ is the so-called {\it critical current} of the Josephson junction, which is the maximum current that it can support without transitioning to a resistive state. $\varphi$ is related to the instantaneous voltage between the electrodes of the junction as
%
\begin{equation}
V = \frac{\Phi_0}{2\pi}\frac{\partial \varphi}{\partial t},
\end{equation}
%
where $\Phi_0 =h/2e \approx 2.05\times 10^{-15}\;\mathrm{Wb}$ is the {\it magnetic flux quantum}. These two simple equations yield a system exhibiting a  wealth of interesting physical phenomena which are used today in various applications such as quantum limited amplifiers \citep{vijay_invited_2009}, generation of Terahertz radiation \citep{ozyuzer_emission_2007} and voltage standards \citep{levinsen_inverse_1977}. The energy associated with the phase difference across the Josephson junction is
%
\begin{equation}
E = E_J\cdot(1-\cos{\varphi})
\end{equation}
%
where $E_J = I_c \Phi_0/2\pi$ is the so-called {\it Josephson energy}. In addition to this Josephson energy, the junction usually has an electrostatic energy associated to its capacitance (formed by its two separated electrodes) given as $E_c = Q^2/2C$, with $Q$ being the charge accumulated on each of the electrodes of the junction.

\smallskip

For currents $I\ll I_c$, the Josephson junction behaves approximately like a nonlinear inductance
%
\begin{equation}
L_J(\varphi) = \frac{\Phi_0}{2\pi I_c \cos{\varphi}} \approx L_{J0}\left[1+\frac{\varphi^2}{2}+\begin{mathcal}O\end{mathcal}(\varphi^4)\right],\label{eq:josephson_inductance}
\end{equation}
%
where $L_{J0}=\Phi_0/2\pi I_c$ is the so-called {\it Josephson inductance}. 

\smallskip

Using the potential and capacitive energies of the Josephson junction, we can formulate the quantum Hamiltonian of the device, which is
%
\begin{equation}
\hat{H} = \frac{1}{2C}\hat{Q}^2+E_J(1-\cos{\hat{\varphi}}),
\end{equation}
%
where $\hat{\varphi}$ and $\hat{Q}$ are now conjugate quantum operators with $[\hat{\varphi},\hat{Q}]=-i\hbar$ that, in analogy to a classical pendulum, play the role of position and momentum for the Josephson junction. In the limit of small angles $\varphi$, the Hamiltonian becomes that of a quantum harmonic LC oscillator. The nonlinearity present in the system is a key ingredient for realizing a Josephson junction qubit since it makes it possible to drive transitions between the first two quantum states of the device without also exciting higher quantum states, as would be the case for a quantum harmonic oscillator.

\subsection{Quantization of Electrical Circuits}

In this section we outline a general method to treat arbitrary electrical circuits as the ones discussed before within the framework of quantum-mechanics, hence {\it quantizing} them. This introduction on circuit quantization presented in this chapter is based on an seminal article by M. Devoret \cite{devoret_quantum_1995}. A more specific example of circuit quantization can be found in \citep{burkard_multilevel_2004}.

\smallskip

An electrical circuit is fully characterized by the parameters of its elements and its topology. The latter can be described as a set of nodes $j$ connected by a number of branches $i$ . In classical circuit theory, each branch is described by a voltage $V_i$ between its ends and a current $I_{i}$ flowing through it. The Kirchhoff laws demand that the sum of the branch voltages $V_i$ along any closed path in the circuit must be zero and that the sum of currents flowing in and out of each node must be zero. For the quantization of electrical circuits it is usually more convenient to replace voltages and currents with branch charges and fluxes that are defined as
%
\begin{eqnarray}
\Phi_i(t) & = & \int\limits_{-\infty}^t V_i(t') dt' ;\\
Q_i(t) & = & \int\limits_{-\infty}^t I_i(t') dt'.
\end{eqnarray}
%
The Kirchhoff laws now write
%
\begin{align}
\sum\limits_{i} Q_i  =  Q_c & & , & & \sum\limits_{i}\Phi_i = \Phi_c \label{eq:kirchhoff_charge}
\end{align}
%
where $Q_c$ and $\Phi_c$ are constants and where the first sum is over charges $Q_i$ of all elements connected to a certain node and the second one is over all branches forming a closed loop in the circuit. We can obtain a complete set of node and branch equations for any given circuit by constructing the so-called {\it spanning tree} of the circuit, which is a tree in which all nodes are connected to an arbitrarily chosen {\it ground node} by one unique path \citep{devoret_quantum_1995}. From the spanning tree we can obtain a complete set of branches and the corresponding Kirchoff equations for the fluxes $\Phi_i$ around them. Together with the set of Kirchhoff equations for the charges at each node, we can use this system of equations to eliminate unnecessary circuit variables and obtain a description of the circuit using a minimal set of degrees of freedom. Now, to quantize a circuit made up of non-dissipative elements we can follow the method given in \cite{yurke_quantum_1984}, writing the Lagrangian (using the reduced set of variables) as 
%
\begin{equation}
\begin{mathcal}L\end{mathcal}(\Phi_1,\hdots,\Phi_n,\dot{\Phi}_1,\hdots,\dot{\Phi}_n) = \sum\limits_i \begin{mathcal}V\end{mathcal}_i - \sum\limits_i \begin{mathcal}T\end{mathcal}_i \label{eq:circuit_lagrangian}
\end{equation}
%
where the sum $i$ runs over all circuit elements and $\begin{mathcal}V\end{mathcal}_i$ and $\begin{mathcal}T\end{mathcal}_i$ are the potential and kinetic energies associated to the $i$-th circuit element. Here, linear inductances contribute only to the potential energy as $\begin{mathcal}V\end{mathcal}_{L_i} = \Phi_i^2/2L_i$, whereas linear capacitances contribute only to the kinetic energy as $\begin{mathcal}T\end{mathcal}_{C_i}=C_i\dot{\Phi}_i^2/2$. Resistors can be described within the Lagrangian formalism by modeling them as semi-infinite transmission lines with a characteristic impedance matching their resistance \citep{yurke_quantum_1984}. We can also include general nonlinear capacitances and inductances that obey the relations $\dot{\Phi}=f_C(Q)$ and $\dot{Q}=g_L(\Phi)$ between their node flux and charge, and whose energies are given as
%
\begin{eqnarray}
E_C = \int\limits_0^Q f_C(Q)dQ \\
E_L = \int\limits_0^\Phi g_L(\Phi)d\Phi
\end{eqnarray}
%
A Josephson junction, for example, can be described as a nonlinear inductance with $g_L^{JJ}(\Phi)=I_c\sin{(2\pi\Phi/\Phi_0)}$, having an associated energy
%
\begin{equation}
E_L^{JJ} = \int\limits_0^\Phi I_c\sin{\left(2\pi\Phi/\Phi_0\right)}=E_J(1-\cos{\left[2\pi\frac{\Phi}{\Phi_0}\right]}),
\end{equation}
%
where $E_J = I_c\Phi_0/2\pi$. Transmission lines can be quantized by a similar approach, as shown e.g. in \citep{yurke_quantum_1984}. Externally imposed charges and fluxes can be modeled as ``pre-charged'' capacitors and inductors with infinite charge or flux and infinite capacitance or inductance that get renormalized at the end of the quantization process \citep{devoret_quantum_1995}. Externally imposed voltages and currents can be treated like this as well by converting them to corresponding fluxes or charges. From the Lagrangian as given by eq. (\ref{eq:circuit_lagrangian}) we can obtain the classical equations of motion of the system by variation of the action
%
\begin{equation}
\frac{\partial}{\partial t}\left( \frac{\partial \begin{mathcal}L\end{mathcal}}{\partial\dot{\Phi}_i}\right)-\frac{\partial \begin{mathcal}L\end{mathcal}}{\partial \Phi_i} = 0
\end{equation}
%
For each flux $\Phi_i$ we obtain its canonically conjugate momentum $Q_i$ by the equation
%
\begin{equation}
Q_i = \frac{\partial \begin{mathcal}L\end{mathcal}}{\partial(\dot{\Phi}_i)}, \label{eq:canonical_momentum}
\end{equation}
%
where $\dot{\Phi}_i=d\Phi_i/dt$. Having obtained $\Phi_i$ and $Q_i$, we can calculate the Hamiltonian $\begin{mathcal}H\end{mathcal}$ of the system by applying the transformation
%
\begin{equation}
\begin{mathcal}H\end{mathcal}(\Phi_1,\hdots,\Phi_n,Q_1,\hdots,Q_n) = \sum\limits_j \dot{\Phi}_i Q_i - \begin{mathcal}L\end{mathcal}(\Phi_1,\hdots,\Phi_n,\dot{\Phi}_1,\hdots,\dot{\Phi}_n) \label{eq:l_to_h}
\end{equation}
%
This Hamiltonian, written in generalized coordinates, yields the full set of equations of motion of the electrical circuit and depends only on the canonically conjugate variables $\Phi_{1},\hdots,\Phi_n$ and $Q_1,\hdots,Q_n$. First Quantization of the circuit can then be done by simply replacing the classical variables by quantum observables such that $\Phi_i\to\hat{\Phi}_i$ and $Q_i\to\hat{Q}_i$ and imposing commutation relations between them:
%
\begin{eqnarray}
\left[\hat{Q}_i(t),\hat{Q}_j(t')\right] & = & 0 \\
\left[\hat{\Phi}_i(t),\hat{\Phi}_j(t') \right] & = & 0 \\
\left[\hat{\Phi}_i(t),\hat{Q}_i(t')\right] & = & -i\hbar\delta_{ij}\delta(t-t') \label{eq:quantization_commutation_relations}
\end{eqnarray}
%
As an example, in the next section we apply this quantization method to to so-called {\it Cooper pair box} circuit, which is highly relevant to this work.

\subsection{The Cooper Pair Box}

\begin{figure}
	\centering
	\includegraphics[width=\textwidth]{"./material/figures/introduction/cooper_pair_box"}
	\caption{a) The circuit schematic of a Cooper Pair Box (CPB). The device consists of a Josephson junction capacitively coupled to a voltage source. The extra capacitance of the Josephson junction is modeled by a capacitor $C_\Sigma$. Charges can accumulate on the island between the voltage source and the Josephson junction. b) A {\it split Cooper pair box}, where instead of one junction two of them are arranged in a loop. With this geometry it is possible to tune the effective Josephson energy of the circuit by changing the magnetic flux inside the junction loop. c) The schematic of a split CPB capacitively coupled to ground. This schematic corresponds most closely to the experimental CPB circuit that we use in this work.}
	\label{fig:cpb_circuit}
\end{figure}

The {\it Cooper pair box (CPB)} is a device containing a Josephson junction coupled to an input voltage source through a gate capacitance $C_g$, as shown in fig. \ref{fig:cpb_circuit}a. Often one also uses two junction in a loop instead of one single one, as shown in fig. \ref{fig:cpb_circuit}b, which allows one to tune the effective Josephson energy of the system by changing the flux inside the junction loop, as will be explained in more detail later. Finally, in our experimental setup one separates the ground electrode of the CPB capacitively from the ground, as shown in fig. \ref{fig:cpb_circuit}c. In this section we discuss only case (a), since as we will show, (b) can be mapped to the simpler circuit (a) and the topology of (c) is mathematically equivalent to that of (b). The simple CPB circuit \ref{fig:cpb_circuit}a consists of  three nodes (including ground) and two branches. The flux $\Phi_2$ is not independent since it is set by the voltage source $V_g$, so we can directly eliminate it from the equations. This leaves us with only one remaining active node $\Phi_1$. Using these definitions, the Lagrangian of the circuit is given as
%
\begin{equation}
\begin{mathcal}L\end{mathcal}=\frac{1}{2}C_\Sigma\dot{\Phi}_1^2+\frac{1}{2}C_g\left(\dot{\Phi}_1+V_g\right)^2-E_J\left(1-\cos{\phi_1}\right)
\end{equation}
%
where $\phi_1=2\pi\Phi_1/\tilde{\Phi}_0$ and where we have written the magnetic flux quantum as $\tilde{\Phi}_0$ to distinguish it from the flux at the ground node. The canonical momentum $Q_1$ associated to the flux $\Phi_1$ is given as
%
\begin{equation}
Q_1 = \frac{\partial \begin{mathcal}L\end{mathcal}}{\partial \dot{\Phi}_1} = C_\Sigma \dot{\Phi}_1+C_g(V_g+\dot{\Phi}_1) \label{eq:q_of_phi}
\end{equation}
%
From this, we can directly calculate the Hamiltonian by using eq. (\ref{eq:l_to_h}) and substituting $Q_1$ as given by eq. (\ref{eq:q_of_phi}) for $\dot{\Phi}_1$, which yields
%
\begin{equation}
\begin{mathcal}H\end{mathcal} = E_J\left(1-\cos{\phi_1}\right)+\frac{(Q_1-C_g V_g)^2}{2(C_\Sigma+C_g)}-\frac{1}{2}C_g V_g^2
\end{equation}
%
Quantization of the Hamiltonian is completed by replacing $Q_i\to \hat{Q}_i$ and $\phi_i\to\hat{\phi}_i$ and imposing the commutation relations given by eqs. (\ref{eq:quantization_commutation_relations}). If, in addition we introduce reduced operators for the charge $\hat{n}=\hat{Q}/2e$ and discard the energy stored in the voltage source (which is irrelevant). We then obtain the Hamiltonian of the Cooper pair box, as formulated e.g. in the thesis of V. Bouchiat \citep{bouchiat_quantum_1998},
%
\begin{equation}
\hat{H} = E_C \left( \hat{n} - n_g\right)^2-E_J \cos{\hat{\phi_1}}, \label{eq:cpb_hamiltonian}
\end{equation}
%
with $E_C = (2e)^2 / (C_\Sigma+C_g)$ the charging energy of the Cooper pair box and $n_g=V_g C_g /2e $ the reduced gate charge.

\smallskip

For the split Cooper pair box as shown in fig. \ref{fig:cpb_circuit}b, the treatment is slightly modified. First of all, we write the Josephson energies of the two junctions as $E_{J1}=(1+d)E_J/2$ and $E_{J2}=(1-d)E_J/2$, where $d$ is the energy asymmetry between the junctions. When imposing an external phase $\phi_{ext}=2\pi\Phi_{ext}/\tilde{\Phi}_0$ in the loop, the potential energy of the two junctions can be written as
%
\begin{eqnarray}
\begin{mathcal}V\end{mathcal}_J & = & -\frac{E_J}{2}\left[(1+d)\cos{(\phi_1-\phi_{ext}/2)}+(1-d)\cos{(\phi_2+\phi_{ext}/2)} \right] \\
& = & -E_J\left[\cos{\phi_1}\cos{\frac{\phi_{ext}}{2}}+d\sin{\phi_1}\sin{\frac{\phi_{ext}}{2}}\right].
\end{eqnarray}
%
The remaining part of the quantization process proceeds as above, yielding a Hamiltonian of the split Cooper pair box of the form
%
\begin{equation}
\hat{H}_{split} = E_C(\hat{n}-n_g)^2-E_J\left[\cos{\hat{\phi}_1}\cos{\frac{\phi_{ext}}{2}}+d\sin{\hat{\phi}_1}\sin{\frac{\phi_{ext}}{2}}\right]
\end{equation}
%
This Hamiltonian can be recast in the form \citep{cottet_implementation_2002}
%
\begin{equation}
\hat{H}_{split} = E_C(\hat{n}-n_g)^2-E_J'(d,\phi_{ext})\cos{[\hat{\phi}_1+\gamma(\phi_{ext})]},
\end{equation}
%
where
%
\begin{eqnarray}
E_J'(d,\phi_{ext}) & = & E_J\sqrt{\frac{1+d^2+(1-d^2)\cos{\phi_{ext}}}{2}} \\
\tan{\gamma(\phi_{ext})} & = & -d\tan{\frac{\phi_{ext}}{2}}
\end{eqnarray}
%
It is therefore possible to map the Hamiltonian of the split CPB to that of the single-junction one by defining $\hat{\theta}\to\hat{\phi}_1+\gamma(\phi_{ext})$ and $E_J\to E_J'(d,\phi_{ext})$. 

Using the Hamiltonian defined in eq. (\ref{eq:cpb_hamiltonian}), we can calculate the wave function of the simple Cooper pair box. The variables $\hat{n}$ and $\theta$ are conjugate such that $[\theta,\hat{n}]=-i\hbar$, the corresponding wave function $\Psi_k(\theta) = \bracket{\theta,k}$ will therefore satisfy a Schrödinger equation of the form
%
\begin{equation}
E_k \Psi_k(\theta) = E_C(\frac{1}{i}\frac{\partial}{\partial \theta}-n_g)^2 \Psi_k(\theta) - E_J \cos{\left(\theta\right)}\Psi_k(\theta) \label{eq:cpb_schroedinger_equation}
\end{equation}
%
Since the potential $E_J\cos{(\theta)}$ is $2\pi$ periodic, the solution will be of the form
%
\begin{equation}
\Psi_k(\theta) = \Psi_k(\theta+2\pi),
\end{equation}
%
which allows us to map eq. (\ref{eq:cpb_schroedinger_equation}) to the so-called {\it Mathieu  equation}
%
\begin{equation}
\frac{d^2y}{dx^2}+\left[a-2q\cos{(2x)}\right]y = 0
\end{equation}
%
The {\it Floquet theorem} states that all solutions to this equation can be written in the form
%
\begin{equation}
F(a,q,x) = \exp{\left(i\mu x\right)}P(a,q,x)
\end{equation}
%
The most general solutions of this equation are given as \citep{cottet_implementation_2002}
%
\begin{equation}
\Psi_k(r,q,\theta) = \mcal{C}_1\exp{\left(i n_g \theta \right)}\mcal{M}_C\left(\frac{4E_k}{E_C},-\frac{2E_J}{E_C},\frac{\theta}{2}\right)+\mcal{C}_2\exp{\left(i n_g \theta \right)}\mcal{M}_S \left(\frac{4 E_k}{E_C},-\frac{2 E_J}{E_C},\frac{\theta}{2}\right)
\end{equation}
%
with 
%
\begin{equation}
E_k = \frac{E_C}{4}\mcal{M}_A \left(r_k,-\frac{2 E_J}{E_C} \right)
\end{equation}
%
Here, $\mcal{M}_C$, $\mcal{M}_S$ are the so-called {\it Mathieu functions} and $\mcal{M}_A$ corresponds to the eigenvalue corresponding to each solution. Following the convention in \citep{cottet_implementation_2002} we order the $E_k$ such that the energy increases with increasing $k$, yielding \citep{koch_charge-insensitive_2007}
%
\begin{eqnarray}
r_k(n_g) & = & \sum\limits_{l\pm 1}\left[\mathrm{int}(n_g+l/2)\mathrm{mod}\;2\right] \notag \\
&  & \times\left\{\mathrm{int}(n_g/2)+l(-1)^k[(k+1)\mathrm{div}\;2]\right\}
\end{eqnarray}
%

\begin{figure}[ht!]
	\includegraphics[width=\textwidth]{"./material/mathematica/cooper_pair_box_energies"}
	\caption{Energies of the first four energy levels of the Cooper pair box for different ratios $E_J/E_C$, plotted as a function of the gate charge $n_g$. As can be seen, for $E_J \gg E_C$, the charge-dispersion curve becomes almost completely flat.}
	\label{fig:CooperPairBoxEnergies}
\end{figure}

We denote the energy differences between individual energy level by $E_{ij} = E_j - E_i$. We also define the absolute and relative anharmonicities of the first two energy levels as $\alpha \equiv E_{12}-E_{01}$ and $\alpha_r \equiv \alpha / E_{01}$. An in-depth treatment of the Cooper pair box can be found e.g. in \citep{cottet_implementation_2002}. Using the basis states $\ket{i}$ of the CPB, we can rewrite its Hamiltonian in the form 
%
\begin{equation}
\hat{H} = \hbar\sum\limits_{i=0} \omega_i\ket{i}\bra{i},
\end{equation}
%
where $\hbar\omega_i$ is the energy associated to the i-th CPB state. Disregarding CPB levels $i \ge 2$, we can also formulate an approximate qubit Hamiltonian of the CPB of the form
%
\begin{equation}
\hat{H} = \frac{\hbar\omega_{01}}{2}\hat{\sigma}_z, \label{eq:cpb_qubit_hamiltonian}
\end{equation}
%
where $\omega_{01}=\omega_1-\omega_0$ is the transition frequency of the $\ket{0}\to\ket{1}$ transition of the CPB.

\subsubsection{The Transmon Qubit}

The Transmon qubit as developed in Yale by R. Schoelkopf {\it et. al.} \cite{koch_charge-insensitive_2007,wallraff_strong_2004} is a Cooper pair box in the regime where $E_J \gg E_C$. As shown in fig. \ref{fig:CooperPairBoxEnergies}, in this regime the charge dispersion of the energy levels of the Cooper pair box becomes almost flat, thus rendering the transition frequency $E_{01}$ practically insensitive to the value of the gate charge $n_g$. This reduced sensitivity to charge noise is highly advantageous in experiments since it increases the coherence time of the qubit. However, when increasing the ratio $E_J/E_C$, we also reduce the anharmonicity $\alpha_r$ of the qubit, therefore limiting the speed of gate operations that can be realized with this system (driving errors related to weak anharmonicity will be discussed more thoroughly chapter \ref{chapter:processor_characterization}). In the limit $E_J \gg E_C$ these qubit anharmonicities are well approximated by $\alpha \simeq -E_C$ and $\alpha_r \simeq -(8 E_J / E_C)^{-1/2}$. However, $\alpha_r$ decreases only geometrically with $E_J/E_C$, whereas the sensitivity of the qubit to charge noise decreases exponentially with the ratio of Josephson and charging energy.

\subsubsection{Decoherence of the Transmon}

In this section we briefly list the most relevant decoherence channels for the Transmon qubit. An in-depth derivation of the decoherence of the CPB and the Transmon can be found e.g. in \citep{cottet_implementation_2002,koch_charge-insensitive_2007}, here we just give the relevant expressions that we use later in this work to estimate the coherence of the Transmon qubits in our quantum processor. Fundamentally, {\it relaxation} and {\it dephasing} are the relevant decoherence mechanisms of the qubit. Following the treatment by Cottet {\it et. al.} \citep{cottet_implementation_2002}, a perturbation of the CPB Hamiltonian can be written as $\delta \hat{H}_{\lambda,S}=-\hbar/2(\mathbf{D}_\lambda\cdot\mathbf{\sigma})\delta \mathbf{\lambda}_S$, where $\mathbf{D}_\lambda$ gives the coupling of the Hamiltonian to a given noise channel $\lambda$. The most relevant couplings for relaxation and dephasing through the charge and current/phase channels of the CPB are
%
\begin{eqnarray}
D_{n_g,z} & = & -2\frac{E_C}{\hbar}\left(\bra{1}\hat{n}\ket{1}-\bra{0}\hat{n}\ket{0}\right) \\
D_{\delta/2\pi,z} & = & \frac{2\pi\Phi_0}{\hbar}\left(\bra{1}\hat{I}\ket{1}-\bra{0}\hat{I}\ket{0}\right) \\
D_{n_g,\perp} & = & 4\frac{E_C}{\hbar}\left|\bra{0}\hat{n}\ket{1}\right| \\
D_{\delta/2\pi,\perp} & = & \frac{4\pi\Phi_0}{\hbar}\left|\bra{0}\hat{I}\ket{1}\right|.
\end{eqnarray}
%
For the relaxation processes,we can calcualte the relaxation rate from the sensitivity $D_{\lambda,\perp}$ and spectral density of the noise channel $S_{\lambda}(\omega_{01})$ using Fermi's golden rule:
%
\begin{equation}
\Gamma^{rel}_{S,\lambda} = \frac{\pi}{2}|D_{\lambda,\perp}|^2 S_{\lambda}(\omega_{01})
\end{equation}
%
Similarly, for the dephasing processes we can calculate the dephasing rates using the corresponding sensitivities and spectral densities as
%
\begin{equation}
\Gamma_{S,\lambda}^\phi = \pi D_{\lambda,z}^2 S_{\lambda}(\omega = 0),
\end{equation}
%
assuming that the noise is regular at low frequencies. In the following paragraphs, we will discuss the most important relaxation and dephasing mechanisms for the CPB.

\paragraph{Relaxation Through the Charge Channel}

Since the CPB is coupled to an external impedance (typically 50 $\Omega$) through a gate capacitance $C_g$, relaxation into free modes of the heat bath represented by this impedance can occur. The spectral density of gate charge fluctuations is given as

%
\begin{equation}
S_{V_g}(\omega) = \frac{\hbar\omega}{2\pi}\left[\coth{\left(\frac{\hbar \omega}{2 k_B T}\right)}\right]\mathrm{Re}\left(Z_g(\omega)\right).
\end{equation}
%
The corresponding matrix element describing relaxation through the gate is given as $\hbar D_{n_{g,\perp}}=4 E_C\bra{0}\hat{h}\ket{1}$. The resulting relaxation rate is hence
%
\begin{equation}
\Gamma_1^{gate} = 16\pi\beta^2 \omega_{01} \frac{\mathrm{Re}\left(Z(\omega_{01})\right)}{R_K}\left|\bra{0}\hat{n}\ket{1}\right|^2,
\end{equation}
%
where $R_K = h/e^2$ and $\beta=C_g/C_\Sigma$.

\paragraph{Relaxation Through the Flux Channel}

The coupling of the CPB to a so-called {\it flux line} --which is a transmission line used for inducing fast flux changes in the CPB junction loop-- can result in additional relaxation of the CPB into the impedance represented by this line. Similar to above, we first formulate the spectral density of magnetic flux fluctuations in the flux line,
%
\begin{equation}
S_{\Phi/\Phi_0}(\omega) = \left(\frac{M}{\Phi_0}\right)^2 S_I(\omega) = \left(\frac{M}{\Phi_0}\right)^2\frac{\hbar\omega}{2\pi}\mathrm{Re}\left(\frac{1}{Z_{coil}(\omega)}\right)\left[\coth{\left(\frac{\hbar\omega}{2k_B T}\right)}+1\right]
\end{equation}
%
The sensitivity of the qubit to flux noise is
%
\begin{equation}
\left|D_{\Phi/\Phi_0,\perp}\right|  = \frac{E_J}{2}\sqrt{1-(1-d^2)\cos^2{\left(\frac{\Phi}{2\Phi_0}\right)}}.
\end{equation}
%
With this, we can calculate the resulting relaxation rate
%
\begin{equation}
\Gamma_1^{fl} = \frac{\pi}{2}S_{\delta \omega,\perp}(\omega_{01})=\frac{\pi}{2}\left|D_{\Phi/\Phi_0,\perp}\right|^2 S_{\Phi/\Phi_0}(\omega_{01})
\end{equation}
%
\paragraph{Dephasing due to Flux Noise}
The noise in the magnetic flux seen by the qubit can induce dephasing. When considering a universal $1/f$-type flux noise with a typical reduced amplitude $A=10^{-5}\Phi_0$ \citep{koch_charge-insensitive_2007}, we obtain a corresponding dephasing rate
%
\begin{equation}
\Gamma_\phi^{\delta \Phi} \propto 3.7A\left|\frac{\partial \omega_{ge}}{\partial \Phi}\right| = 3.7\frac{\pi A}{\hbar \Phi_0}\sqrt{2E_C(E_{J1}+E_{J2})\left|\sin{\left(\frac{\pi\Phi}{\Phi_0}\right)}\tan{\left(\frac{\pi\Phi}{\Phi_0}\right)}\right|}
\end{equation}
%

\paragraph{Dephasing due to Charge Noise}

The sensitivity of the CPB to charge noise is given as
%
\begin{equation}
\Gamma_\phi^{\delta n_g} \simeq 3.7A \left|\frac{\delta \omega_{01}}{\delta n_g}\right|.
\end{equation}
%
In the limit $E_J\gg E_C$ this expression yields
%
\begin{equation}
\Gamma_\phi^{\delta N_g} \simeq 3.7\frac{A\pi}{\hbar}\left| (\epsilon_1-\epsilon_0)\sin{\left(2\pi N_g\right)}\right| \leq 3.7\frac{A\pi}{\hbar}|\epsilon_1|,
\end{equation}
%
where $\epsilon_1$ is the modulation amplitude of the first excited CPB level \citep{koch_charge-insensitive_2007}:
%
\begin{equation}
\epsilon_m\simeq (-1)^m E_C\frac{2^{4m+3}}{m!}\sqrt{\frac{2}{\pi}}\left(\frac{2E_J}{E_C}\right)^{\frac{m}{2}+\frac{3}{4}}\exp{\left(-\sqrt{\frac{32E_J}{E_C}}\right)}.
\end{equation}
%
As can be seen, the sensitivity of the CPB to charge noise decreases exponentially with the ratio $E_J/E_C$ and for typical values that we use in this work ($E_J/E_C\approx 10$), we obtain dephasing times in the seconds range, hence we usually ignore this dephasing channel. 

\subsection{Coplanar Waveguide Resonators}

Another circuit element that we will encounter many times in this work is the so-called {\it coplanar waveguide resonator} (CPW). A coplanar waveguide is a flat structure with a central conductor that is separated by a gap from a ground plane on either side. In general, it can be treated as a so-called {\it transmission line}. A detailed treatment of the physics of transmission lines can be found e.g. in \cite{pozar_microwave_2011}. The equation that describes the propagation of an electromagnetic wave along the extended dimension $z$ of the waveguide is
%
\begin{eqnarray}
V(z,t) & = & \exp{\left(i\omega t\right)}\cdot\left[V^+ \exp{\left(-i\gamma z\right)}+V^-\exp{\left(i\gamma z\right)}\right] \\
I(z,t) & = & \frac{1}{Z_0}\exp{\left(i\omega t\right)}\cdot\left[V^+ \exp{\left(-i\gamma z\right)}-V^-\exp{\left(i\gamma z\right)}\right].
\end{eqnarray}
%
Here, $\gamma = \alpha+i\beta = \sqrt{(R+i\omega L)(G+i\omega C)}$ is the so-called {\it propagation constant} which describes the dispersion and damping of electromagnetic waves along the waveguide, $\omega$ is the angular frequency of the electromagnetic wave and $L$, $C$, $R$ and $G$ are the characteristic inductance, capacitance, resistance and conductance of the transmission line (For a lossless line $G=R=0$). The voltages $V^+$ and $V^-$ correspond to waves traveling in different directions along the waveguide.

\smallskip

If we regard now a waveguide of finite length $l$, we can model the voltages and currents at both ends as \citep{pozar_microwave_2011}
%
\begin{equation}
\left( \begin{array}{c} V_1 \\ I_1 \end{array}\right) = \left( 
		\begin{array}{cc}
			\cos{\gamma l} & iZ_r \cos{\gamma l} \\
			i Y_r \sin{\gamma l} & \cos{\gamma l}
		\end{array}
		\right) \cdot \left(
		\begin{array}{c}
			V_2 \\ I_2
		\end{array}
		\right), \label{eq:cpw_abcd_matrix}
\end{equation}
%
where $Z_r=\sqrt{L/C}$ is the impedance of the waveguide and $Y_r=1/Z_r$ the corresponding admittance. One can easily create a resonator using such a finite-length coplanar waveguide. Here, we consider the open-ended $\lambda / 2$ CPW resonator that we use in our experiments to realize the qubit readout resonator, as shown in fig. \ref{fig:lambda_over_2_response}b-c. To realize the resonator, we terminate the transmission line at one end by an open gap and connect the other end to a drive line through an input capacitance $C_{in}$. We can make use of eq. (\ref{eq:cpw_abcd_matrix}) to calculate the end voltages and currents of the resonator,demanding that $I_2=0$ (since the resonator is open-ended, see fig. \ref{fig:lambda_over_2_response}a). We obtain for the voltage $V_1$ and current $I_2$ the relation
%
\begin{eqnarray}
V_1 & = & \cos{\gamma l} V_2 \\
I_1 & = & i Y_r \sin{\gamma l} V_2
\end{eqnarray}
%
The impedance of the resonator is given as $V_1/I_1 = -i Z_r \cot{\gamma l}$. The input impedance of the resonator as seen through the gate capacitance $C_{in}$ is then
%
\begin{equation}
Z_{in} = -i \left(Z_r \cot{\gamma l}+\frac{1}{\omega C_{in}}\right) \label{eq:cpw_impedance}
\end{equation}
%
The $S_{11}$ reflection coefficient of the resonator when coupling it to an input line with impedance $Z_0$ is
%
\begin{equation}
S_{11} = \frac{Z_{in}-Z_0}{Z_{in}+Z_0} = \frac{i(Z_r\cot{\gamma l}+1/(\omega C_{in}))-Z_0}{Z_0+i(Z_r\cot{\gamma l}-1/(\omega C_{in}))}
\end{equation}
%
Now, when measuring the reflection of an incoming signal with voltage $V^+$ at frequency $2\pi f=\omega$ and phase $\phi_0$, the phase of the reflected signal $\phi_{ref}$ will be simply given as $\phi_{ref}-\phi_0=\mathrm{Arg}[V^-/V^+] = \mathrm{Arg}[S_{11}]$. The amplitude of the reflected signal $|S_{11}|$ is always unity since we assumed that there are no internal losses inside the resonator. Fig \ref{fig:lambda_over_2_response}d shows this phase along with the absolute value of the input impedance $|Z_{in}|$ for an exemplary $\lambda/2$ resonator, shown in reduced units of $[l/v]$, with impedances $Z_r=Z_0=50\;\mathrm{\Omega}$, $\alpha=0$ and $C_{in}=10^{-3}/\omega_r\;[\mathrm{Hz}\cdot \mathrm{F}]$, where $\omega_r = 2\pi f_r$ is the angular resonance frequency of the resonator.

\begin{SCfigure}
	\includegraphics[width=10cm]{"./material/mathematica/cpw_lambda_over_4_phase_and_z_with_schematic"}
	\caption{a) The electric field $E$ and current $I$ inside a $\lambda/2$ resonator. b) The circuit model of an open $\lambda/2$ resonator capacitively coupled to a drive circuit. The resonator can be modeled as a series of infinetesimal sections of $LC$ elements, as shown in the inset. c) The schematic of a coplanar waveguide (CPW) resonator, showing the electric field $E$, magnetic field $B$ and the current $I$ in the resonator. d) The reflected phase and absolute value of the input impedance of a $\lambda/2$ resonator with $Z_r=Z_0=50\;\mathrm{\Omega}$, $\alpha=0$ and $C_{in}=10^{-3}/\omega_r\;[\mathrm{Hz}\cdot\mathrm{F}]$, plotted as a function of the reduced frequency.}
	\label{fig:lambda_over_2_response}
\end{SCfigure}

We can approximately model the distributed CPW resonator as a lumped element $LC$ resonator, which is useful for e.g. calculating its quality factor. If we regard the input impedance of the resonator in the vicinity of $\omega_r$ such that $\Delta \omega = \omega -\omega_r \ll \omega_r$ and $\beta l = \pi +\pi\Delta \omega /\omega_r$, we obtain an effective impedance
%
\begin{equation}
Z_{in} = -i\frac{Z_{r}}{(\Delta \omega \pi / \omega_r)}
\end{equation}
%
We can identify the quantities in this equation with the input impedance of a parallel LC-resonator, which is approximately given as
%
\begin{equation}
Z_{in} = \frac{Z_0}{1+2i Q \Delta \omega / \omega_r}
\end{equation}
%
with $Q=\omega_r RC$. This yields an effective inductance and capacitance for the transmission line resonator of
%
\begin{eqnarray}
L_{r} & = & \frac{2 Z_r}{\omega_r \pi} \\
C_{r} & = & \frac{\pi}{2\omega_r Z_r}
\end{eqnarray}
%
If the resonator is not lossless, we can also define an effective resistance $R_r=2 Z_r Q/\pi$ that depends on the impedance $Z_r$ and the quality factor $Q$ of the resonator. When coupling this resonator to an input transmission line of impedance $Z_0$ through a  capacitance $C_{in}$ as before, the quality factor of the coupled (or {\it loaded}) resonator will be given as
%
\begin{equation}
Q_L = \omega_r^* \frac{C_r+C^*}{1/R_{r}+1/R^*}
\end{equation}
%
where we have introduced an effective resistance, capacitance and resonance frequency given as
%
\begin{eqnarray}
R^* & = & \frac{1+\omega_r^2 C_{in}^2 Z_0^2}{\omega_r^2 C_{in}^2 Z_0} \\
C^* & = & \frac{C_{in}}{1+\omega_r^2 C_{in}^2 Z_0^2} \\
\omega_r^* & = & \frac{1}{\sqrt{L_r(C_r+C_{in})}}
\end{eqnarray}
%
We make use of these relations and the fact that the external quality factor can be tuned by the value of the gate capacitance $C_{in}$ when designing the qubit readout resonator. 

\subsubsection{Quantization of the resonator}

We can quantize the coplanar waveguide resonator by following the recipe given in the last section. A detailed review of this quantization process can be found e.g. in \citep{yurke_quantum_1984}. If we are not interested in the internal mode structure of the resonator, it suffices to quantize the simple LC model of the CPW resonator. The Hamiltonian of the LC series resonator is
%
\begin{equation}
H = \frac{1}{2C}Q^2+\frac{1}{2L}\Phi^2
\end{equation}
%
It is completely equivalent to a harmonic oscillator, the quantum-mechanical Hamiltonian is therefore given as
%
\begin{equation}
\hat{H} = \omega\hbar\left(\hat{a}^\dagger\hat{a}+\frac{1}{2}\right)
\end{equation}
%
Here, $\omega=\sqrt{1/LC}$ and $\hat{a}^\dagger$ and $\hat{a}$ are so-called {\it creation and annihilation operators} that can be written in function of the flux $\hat{\Phi}$ and charge $\hat{Q}$ operators as
%
\begin{eqnarray}
\hat{a}^\dagger & = & \sqrt{\frac{1}{2\hbar L \omega}}\left(\hat{\Phi}+iL\omega \hat{Q}\right)\\
\hat{a} & = & \sqrt{\frac{1}{2\hbar L\omega}}\left( \hat{\Phi}-iL\omega \hat{Q}\right)
\end{eqnarray}
%
By inverting these relations we obtain the flux and charge operators $\hat{\Phi}$ and $\hat{Q}$. Taking their time derivative gives us the voltage and current operators $\hat{V}$ and $\hat{I}$. These four operators are
%
\begin{eqnarray}
\hat{\Phi} & = & \sqrt{\frac{\hbar}{2C\omega}}\left(\hat{a}^\dagger+\hat{a}\right) \\
\hat{Q} & = & i\sqrt{\frac{C\omega\hbar}{2}}\left(\hat{a}^\dagger-\hat{a}\right) \\
\hat{V} & = & \sqrt{\frac{\hbar\omega}{2C}}\left(\hat{a}^\dagger+\hat{a}\right) \\
\hat{I} & = & i\omega\sqrt{\frac{C\omega\hbar}{2}}\left(\hat{a}^\dagger-\hat{a}\right)
\end{eqnarray}
%
 
\section{Circuit Quantum Electrodynamics}

\begin{SCfigure}
	\includegraphics[width=11cm]{"./material/figures/introduction/cqed/cqed"}
	\caption{a) Images of a $\lambda/2$ resonator, a Transmon qubit and its Josephson junctions. b) The equivalent circuit of the setup. c) The equivalent circuit using a lumped-element model for the $\lambda/2$ resonator.}
	\label{fig:CQED}
\end{SCfigure}

In this section, we investigate the physics of a Transmon qubit that is coupled to a coplanar waveguide resonator. The corresponding research field is today referred to as {\it circuit quantum electrodynamics}, in analogy to {\it cavity quantum electrodynamics} which investigates the physics of Rydberg atoms interacting with a microwave cavity. The qubit-resonator system can be represented as in fig. \ref{fig:CQED}. There, a Transmon qubit is capacitively coupled to a $\lambda/2$ resonator which itself is capacitively coupled to an input transmission line. The $\lambda/2$ resonator can be modeled as an harmonic oscillator when neglecting the internal mode structure, yielding the Hamiltonian
%
\begin{equation}
\hat{H}_r = \hbar(\omega_r+\frac{1}{2})\hat{a}^\dagger\hat{a}.
\end{equation}
%
Here, $\omega_r = 1/\sqrt{L_r C_r}$ gives the resonator frequency of the resonator. On the other hand, the Hamiltonian of the Transmon qubit can be written as a function of its basis states $\ket{i}$ as
%
\begin{equation}
\hat{H}_{q} = \hbar\sum\limits_i \omega_i \ket{i}\bra{i}
\end{equation}
%
where $\hbar\omega_i$ is the energy of the i-th level of the Transmon (and not the transition energy between different states). Due to the capacitance between the qubit and the resonator, a coupling energy between the two arises. For small couplings $C_g \ll C_{in},C_r,C_\Sigma$ , we can estimate this coupling energy simply as
%
\begin{equation}
\hat{H}_{rq} = \frac{1}{2}C_{g}\hat{V}_{qr}^2 = \frac{1}{2}C_g\left(V^0_{rms}(a^\dagger+a)-\hat{V}\right)^2, \label{eq:cqed_coupling}
\end{equation}
%
where $\hat{V}_{qr}$ is the voltage between the coupling capacitance $C_g$, $\hat{V}=2e/C_\Sigma \cdot(n_g-\hat{n})$ is the voltage between the Transmon electrodes and $V^0_{rms} = \sqrt{\hbar \omega_r/2C_r}$ is the root mean square voltage in the resonator. A rigorous treatment of the coupling energy, which is necessary for large coupling capacitances $C_{qr}\simeq C_{r},C_\Sigma$, would require a full quantization of the coupled qubit-resonator circuit, as performed e.g. in \citep{nguyen_cooper_2008}. The coupling energy in eq. (\ref{eq:cqed_coupling}) can be rewritten as
%
\begin{eqnarray}
\hat{H}_{rq} & = & \frac{1}{2}C_g\left[V_{rms}^0(a^\dagger+a)-\frac{2e}{C_\Sigma}\left(n_g-\hat{n}\right)\right]^2 \notag \\
       & = & 2e \beta V_{rms}^0\hat{n}(a^\dagger+a) + \hdots \label{eq:cqed_coupling}
\end{eqnarray}
%
in the limit $\beta \ll C_\Sigma$, where we defined $\beta = C_g/C_\Sigma$. The terms omitted in eq. (\ref{eq:cqed_coupling}) correspond to energy shifts of the qubit and the resonator which are not directly relevant for the coupling between them. 
 In the limit where the resonator capacity $C_r \gg C_\Sigma$, we can write the effective Hamiltonian of the qubit-resonator system using the uncoupled basis states $\ket{i}$ of the Transmon as
%
\begin{equation}
\hat{H} = \hbar \sum\limits_{j=0} \omega_j \ket{j}\bra{j} + \hbar \omega_r \hat{a}^\dagger \hat{a} + \hbar \sum\limits_{i\ne j} g_{ij} \ket{i}\bra{j}(\hat{a}+\hat{a}^\dagger) \label{eq:cqed_hamiltonian}
\end{equation}
%
Here, the coupling energies $g_{ij}$ are given as
%
\begin{equation}
\hbar g_{ij} = 2\beta e V_{rms}^0 \bra{i}\hat{n}\ket{j} = \hbar g_{ji}^*
\end{equation}
%
When the coupling between the resonator and the Transmon is weak, such that $g_{ij} \ll \omega_r,E_{01}/h$, we can ignore the terms in eq. (\ref{eq:cqed_hamiltonian}) that describe simultaneous excitation or de-excitation of the Transmon and the resonator and obtain the so-called {\it rotating wave approximation}, which is given as
%
\begin{equation}
\hat{H} = \hbar \sum\limits_{j=0} \omega_j \ket{j}\bra{j}+\hbar \omega_r \hat{a}^\dagger \hat{a} + \hbar \sum\limits_{i=0} g_{i,i+1}\left(\ket{i}\bra{i+1}\hat{a}^\dagger +\ket{i+1}\bra{i}\hat{a}\right) \label{eq:cqed_rotating_wave}
\end{equation}
%
This Hamiltonian describes thus a multi-level quantum system coupled to a resonator through a capacitive interaction. The first two terms correspond to the energies of the n-level system and the resonator, respectively. The term $\ket{i}\bra{i+1}\hat{a}^\dagger$ describes the creation of a photon in the resonator accompanied by the de-excitation of the n-level system by one energy level and the term $\ket{i+1}\bra{i}\hat{a}$ describes the opposite process.

\subsection{Dispersive Limit \& Qubit Readout}

When the qubit frequency is far detuned from the resonator frequency such that $|\omega_{ij}-\omega_r| \gg g_{ij}$, direct qubit-resonator interactions are almost completely suppressed and only a dispersive shift of the transition frequency of both systems remains as an effect of the coupling between them. This effect has been discussed e.g. in \cite{koch_charge-insensitive_2007} and yields an effective Hamiltonian of the form
%
\begin{equation}
\hat{H}_{eff} = \frac{\hbar\omega_{01}'}{2}\hat{\sigma}_z+(\hbar\omega_r'+\hbar \chi \hat{\sigma}_z)\hat{a}^\dagger \hat{a},
\end{equation}
%
where we have used the two-level qubit Hamiltonian as given by eq. (\ref{eq:cpb_qubit_hamiltonian}). Here, the resonance frequencies of the qubit and the resonator are shifted as $\omega_{01}'=\omega_{01}+\chi_{01}$ and $\omega_r' = \omega_{r}-\chi_{12}/2$ and the dispersive shift is given as $\chi=\chi_{01}-\chi_{12}/2$, where $\chi_{ij}=g_{ij}^2/(\omega_{ij}-\omega_r)$. As can be seen, for a state with $n$ photons, the energy difference between the two qubit levels is given as
%
\begin{equation}
\omega_{01}^n = \omega_{01}'+2\chi n
\end{equation}
%
Thus, there is a dispersive shift of the qubit transition frequency that is proportional to the number of photons in the resonator. Likewise, the resonance frequency of the resonator gets also shifted by $2\hbar\chi$ depending on the state of the qubit. The latter effect is very useful since it allows us to read out the state of the qubit by measuring the state-dependent frequency displacement of the resonator, as will be explained later.
\subsection{Qubit-Qubit Interaction}

In this section we discuss possible qubit-qubit coupling schemes. We regard a direct coupling scheme involving a capacitive coupling between two qubits and an indirect scheme involving the coupling of multiple qubits to a resonator which acts as a ``quantum bus''.

\subsubsection{Coupling Bus}

For this particular coupling scheme, we consider two (or more) Transmon qubits coupled to the same resonator. Blais {\it et. al.} \citep{blais_quantum-information_2007} showed that extending the single-qubit rotating-wave Hamiltonian as given in eq. (\ref{eq:cqed_rotating_wave}) to this case of two qubits coupled to a resonator yields an effective qubit-qubit coupling Hamiltonian of the form
%
\begin{eqnarray}
\hat{H}_{2q} & = & \hbar\frac{g_1 g_2(\Delta_1+\Delta_2)}{2\Delta_1\Delta_2}(\sigma_1^+\sigma_2^-+\sigma_1^-\sigma_2^+) \label{eq:cqed_bus_coupling}
\end{eqnarray}
%
This approximation is valid in the limit of large qubit-resonator detuning where $\Delta_1 \gg g_1,\Delta_2 \gg g_2$. Here $\Delta_{1,2} = \omega_{01}^{1,2}-\omega_r$ is the detuning of the $\ket{0}\to\ket{1}$ transition frequency of each qubit to the bus resonator. Full energy-exchange between the qubits is achieved when the qubit frequencies are in resonance. By detuning the qubits from the resonator, the effective coupling constant can be varied, which is advantageous in many settings. 

\subsubsection{Direct Capacitive Coupling}

A direct capacitive coupling $C_{qq}$ between two qubits yields a coupling Hamiltonian of the form
%
\begin{eqnarray}
\hat{H}_{qq} & = & \frac{1}{2}C_{qq}\hat{V}_{qq}^2 = \frac{1}{2}C_{qq}\left[\frac{2e}{C_{\Sigma 1}}(n_{g1}-\hat{n}_1)-\frac{2e}{C_{\Sigma 2}}(n_{g2}-\hat{n}_2)\right]^2 \\
& = & \frac{4e^2 C_{qq}}{C_{\Sigma 1}C_{\Sigma_2}}\hat{n}_1\hat{n}_2+\hdots \label{eq:cqed_capacitive_coupling}
\end{eqnarray}
%
Again, this equation is valid in the limit where $C_{qq} \ll C_{\Sigma 1},C_{\Sigma 2}$, for larger capacitances $C_{qq}$ the coupling gets renormalized by a factor $\alpha = 1/(1-C_{qq}^2/[C_{\Sigma 1}C_{\Sigma 2}])$ \citep{nguyen_cooper_2008}. Rewriting this coupling in the basis of uncoupled qubit states yields the effective Hamiltonian
%
\begin{equation}
\hat{H}_{qq} = \hbar 2 g_{qq}\left(\sigma^+_1\sigma^-_2+\sigma^-_1\sigma^+_2\right), \label{eq:cqed_qubit_interaction_hamiltonian}
\end{equation}
%
where $\sigma^+=\ket{1}\bra{0}$ and $\sigma^-=\ket{0}\bra{1}$ and $\sigma_1^\pm=\sigma^\pm\otimes \mathrm{I}$, $\sigma_2^\pm = \mathrm{I}\otimes \sigma^\pm$ and where we have defined the effective qubit-qubit coupling as $\hbar g_{qq} = 2e^2 C_{qq}/C_{\Sigma 1}C_{\Sigma 2}$. Full energy exchange between the qubits is achieved when the qubit frequencies are in resonance. For the more general case of two coupled n-level Transmons, the coupling Hamiltonian takes a slightly more complicated form, as discusses in the Appendix of this thesis. Directly coupling two qubits can be advantageous since it simplifies the circuit layout and does not require an auxiliary quantum system. However, since the coupling is always turned on it is difficult to achieve a sufficiently good ON/OFF ratio that is needed for many applications (e.g. to realize a quantum gate). When the two qubits are in resonance, the time evolution operator of the Hamiltonian in eq. (\ref{eq:cqed_qubit_interaction_hamiltonian}) yields a swapping interaction of the form

%
\begin{equation}
U(t)=\left(\begin{array}{cccc}
1 & 0 & 0 & 0\\
0 & \cos{2\pi tg} & i\sin{2\pi tg} & 0\\
0 & i\sin{2\pi tg} & \cos{2\pi tg} & 0\\
0 & 0 & 0 & 1\end{array}\right)_{\left\{ \left|00\right\rangle ,\left|01\right\rangle ,\left|10\right\rangle ,\left|11\right\rangle \right\} } \label{eq:swap_evolution_operator}
\end{equation}
%
Using this interaction, it is straightforward to implement e.g. an $\sqrt{i\mathrm{SWAP}}$ or $i\mathrm{SWAP}$ quantum gate.
\section{The Josephson Bifurcation Amplifier}

\begin{SCfigure}[1][ht!]
	\includegraphics[width=8cm]{"./material/figures/introduction/nonlinear resonator"}
	\caption{a)The circuit model of a cavity Josephson bifurcation amplifier (CJBA), consisting of two $\lambda/4$ transmission lines joined by a Josephson junction. b) The lumped element circuit model of the (C)JBA, consisting of a capacitor, an inductor, a voltage source and  a Josephson junction that can itself be modeled as a nonlinear inductance.}
	\label{fig:jba_schematic}
\end{SCfigure}

In this section we discuss the physics of superconducting nonlinear bifurcation amplifiers, which we use to realize a single-shot readout scheme for our qubits. Most notably, we discuss the so-called {\it Josephson bifurcation amplifier (JBA)} and the so-called {\it cavity Josephson bifurcation amplifier (CJBA)}, as shown in fig. \ref{fig:jba_schematic}. The CJBA, on the other hand, consists of a transmission line resonator with a Josephson junction embedded in its central conductor. As shown before, a transmission line resonator can be treated mathematically as a lumped elements resonator, hence the physics of the CJBA can be mapped to that of the JBA. We therefore restrict the following discussion in this section to the JBA, noting that the results can be easily applied to the CJBA as well. A more detailed comparison between the JBA and CJBA can be found e.g. in \cite{palacios-laloy_superconducting_2010}. For illustration, fig. \ref{fig:cba_schematic} shows the CJBA readout that we use for our two-qubit experiments. 


\begin{SCfigure}[1][htb!]
	\includegraphics[width=10cm]{"./material/figures/introduction/jba"}
	\caption{The Cavity Josephson Bifurcation (CJBA) readout used in this work. The CJBA consists of two $\lambda/4$ transmission lines connected by a Josephson junction and capacitively coupled to both the qubit and the input transmission line.}
	\label{fig:cba_schematic}
\end{SCfigure}

The circuit in fig. \ref{fig:jba_schematic}b can be modeled classically as
%
\begin{equation}
[L_e+L_J (i)]\ddot{q}+R_e \dot{q}+\frac{q}{C_e} = V \cos{\left(\omega_m t\right)}
\end{equation}
%
Here, $L_J$ the Josephson inductance as given by eq. (\ref{eq:josephson_inductance}) and $L_e$, $C_e$ and $R_e$ are the resistance, inductance and capacitance in series to the Josephson junction in the circuit. $V$ is the amplitude of the driving voltage, which oscillates at a frequency $\omega_m$. Expanding the Josephson inductance in this equation to second order leads to the equation
%
\begin{equation}
\left(L_e+L_J\left[1+\frac{\dot{q}^2}{2 I_0^2}\right]\right)\ddot{q}+R_e \dot{q}+\frac{q}{C_e} = V \cos{\left( \omega_m t\right)}
\end{equation}
%
Defining the total inductance $L_t = L_e+L_J$, the participation ratio $p=L_J/L_t$, the resonance frequency $\omega_r = 1/\sqrt{L_t C_e}$ and the quality factor $Q = \omega_r L_t / R_e$ we can rewrite this as

\begin{equation}
\ddot{q}+\frac{\omega_r}{Q}\dot{q}+\omega_r^2 q + \frac{p \dot{q}^2 \ddot{q}}{2 I_0} = \frac{V}{L_t}\cos{\left(\omega_m t \right)}
\end{equation}

Introducing the reduced variables $\beta = (V/\phi_0 \omega_m)^2(pQ/2\Omega)^3$, $\Delta_m = \omega_r-\omega_m$, $\tau = t\Delta_m$, $\Omega=2Q\Delta_m/\omega_r$ and $u(t) = \sqrt{pQ/2\Omega}\cdot q(t)\omega_m/I_0$ we can further rewrite this equation to obtain
%
\begin{align}
\frac{\Delta_m}{\omega_m}\ddot{u}+\left(\frac{1}{Q \omega_m}+2i\right)\dot{u} \notag \\
 + \left[ 2\left(\frac{\omega_r^2-\omega_m^2}{2\omega_m \Delta_m}\right)+\frac{1}{Q\Delta m}-2|u|^2\right]u & =  2\sqrt{\beta} \label{eq:jba_theory_1}
\end{align}
%
where $\dot{u}=du/d\tau$. In the limit where $Q \gg 1$, $\Delta_m\omega_m \ll 1$ such that $\omega_m+\omega_r \approx 2\omega_m$ and $\ddot{u}\ll\omega_m \dot{u}$ we can simplify this equation to obtain
%
\begin{equation}
\dot{u} = -\frac{u}{\Omega}-iu\left(|u|^2-1\right)-i\sqrt{\beta} \label{eq:jba_reduced_equation}
\end{equation}
%
The stationary solutions of this equation, corresponding to steady-state oscillations, are given as
%
\begin{equation}
\frac{|u|^2}{\Omega^2}+|u|^2\left(|u|^2-1\right)^2 = \beta(\Omega)
\end{equation}
%
This equation can have one or two stable solutions, depending on the parameter $\beta(\Omega)$. The region where multiple solutions exist is usually called the {\it bi-stability region} and is limited by the the two parameter boundaries
%
\begin{equation}
\beta^\pm(\Omega) = \frac{2}{27}\left[1+\left(\frac{3}{\Omega}\right)^2\pm\left(1-\frac{3}{\Omega^2}\right)^{3/2}\right] \label{eq:jba_beta}
\end{equation}
%
Fig. \ref{fig:jba_curves} shows the values of $\beta^\pm(\Omega)$ as well as the phase $\phi\simeq \mathrm{arg}(u(t))$ and the amplitude $\sqrt{B}\propto A \simeq |u(t)|$ of different stable solutions of eq. (\ref{eq:jba_reduced_equation}) for a JBA with $\alpha=0.05$ and $\gamma=0.01$, where $\alpha=Vp^{3/2}/\phi_0 \omega_r$ and $\gamma = R_e/L_t\omega_r$ characterize the drive strength and dissipation in the JBA, respectively. In chapter \ref{chapter:processor_characterization} we show how to use the bistable behavior of the CJBA to construct a single-shot qubit readout.

\begin{SCfigure}
	\includegraphics[width=9cm]{"./material/mathematica/jba_curves"}
	\caption{a)The bi-stability boundaries $\beta^\pm$ of a JBA, plotted as a function of $\Omega/\Omega_c$. b/c)The phase $\phi=\mathrm{arg}(u(t))$ and amplitude $A\simeq |u(t)|$ of different solutions of eq. (\ref{eq:jba_reduced_equation}) for the JBA parameters $\alpha=0.05$ and $\gamma=0.01$. In the hysteretic region, three solutions exists, two of which are stable. The solution realized at a given moment depends thus on the history of the system, allowing for hysteretic behavior.}
	\label{fig:jba_curves}
\end{SCfigure}
