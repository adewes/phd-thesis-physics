\chapter{Theoretical Foundations}

The goal of this chapter is to provide the theoretical foundations needed to interpret and analyze the experiments discussed in the following chapters. We will therefore briefly introduce some basic concepts of quantum mechanics and quantum computing, discuss Transmon qubits and circuit quantum electrodynamics (CQED) and introduce the reader to the Josephson bifurcation amplifier that we use to read out the qubit state in our experiments. Further details on all the elements discussed here will be provided in the relevant sections of the ``Experiments'' chapter.

\section{Clasical \& Quantum Information Processing}

\begin{SCfigure}
	\includegraphics[width=8cm]{"./material/figures/introduction/bloch_sphere"}
	\caption{The Bloch sphere representation of a qubit state $\ket{\psi} = \cos{\frac{\theta}{2}}\ket{0}+e^{i\phi}\sin{\frac{\theta}{2}}\ket{1}$. The state $\ket{\psi}$ is fully characterized by specifying its ``altitude'' and ``azimuth'' angles $\theta$ and $\phi$. Pure quantum states will always lie on the surface of the Bloch sphere, whereas mixed quantum states can also lie anywhere inside the sphere.}
	\label{fig:BlochSphere}
\end{SCfigure}

By definition, computing designates the activity of using computer hardware and software to process information, or {\it data}. Classical information processing can be divided in so-called {\it analog and digital information processing}, the former being based on continously changeable physical quantities whereas the latter is based on incrementally changeable quantities. The fundamental unit of digital information processing is the so-called {\it bit}, which represents a boolean (true/false) information. The discipline of theoretical computer science has been created to investigate the fundamental limits and properties of classical information processing. One of the main foundational theorems of theoretical computer science is the so-called {\it Church-Turing thesis} which provides a universal computing model by saying (basically) that everything which is computable can be efficiently computed using a {\it Turing machine}. Such a Turing machine, in turn, is a simple theoretical devices which is able to run programs that operates on a discrete set of data using a well-specified set of operations. The Turing machine is universal in the sense that any other classical computing device can be efficiently simulated using a Turing machine with the appropriate program and data.

\smallskip

Surprisingly, Richard Feynman discovered in the early 1980 that a classical Turing machine as described above would be unable to efficiently simulate a quantum-mechanical system \citep{feynman_simulating_1982}. A few years later, David Deutsch took up Feynman's idea and developed an information processing framework based on quantum mechanics. He showed that by using this so-called {\it quantum computing} or {\it quantum information processing} framwork one could solve certain problems faster than would be possible with any classical Turing machine \citep{deutsch_quantum_1985}. The work by Feynman and Deutsch created a large interest in the physics community and led to a huge experimental and theoretical effort aimed at realizing a working quantum computer. 

\section{Principles of Quantum Computing}

\subsection{Quantum Bits}

Similar to classical computing, in quantum computing one can define a fundamental unit of information, the so called {\it quantum bit} or {\it qubit}. Such a qubit is a quantum-mechanical two-level system described by the wavefunction
%
\begin{equation}
\ket{\psi} = \cos{\frac{\theta}{2}}\ket{0}+e^{i\phi}\sin{\frac{\theta}{2}}\ket{1}
\end{equation}
%
As can be seen, the state of such a qubit can be described by a pair of real numbers $\theta$ and $\phi$ that characterize the occupation probability of each of the two basis states $\ket{0}$ and $\ket{1}$ and the phase between them. A useful and intuitive representation of such a single-qubit state is the so-called {\it Bloch sphere representation} of a quantum state, which is shown in fig. \ref{fig:BlochSphere}. In this representation, the state $\ket{\psi}$ is located on a unit sphere. The north and south poles of this sphere correspond to the qubit states $\ket{0}$ and $\ket{0}$. All states lying between those two correspond to superposition states, which are characterized by their ``altitude'' and ``azimuth'' angles $\theta$ and $\phi$.

\subsection{Quantum Gates}

Analogously to classical information processing it is necessary to define {\it quantum gates} which act on individual or multiple qubits and allow us to process information with them. In the most general sense, a quantum gate is a unitary quantum operator that acts on one or several qubits. Theoretically there is an infinite number of possible quantum gates, however in order to describe all possible quantum operations that can be performed on a register of qubits it is sufficient to defined an {\it universal set of quantum gates}. Such a universal gate set that will be especially relevant to this work consists of the three single-qubit rotation matrices
%
\begin{eqnarray}
   R_x(\theta)  & = & \left( \begin{array}{cc} \cos{\theta/2} & -i\sin{\theta/2} \\ -i\sin{\theta/2}  & \cos{\theta/2} \end{array} \right) \\ 
   R_y(\theta)  & = & \left( \begin{array}{cc} \cos{\theta/2} & -\sin{\theta/2} \\ +\sin{\theta/2} & \cos{\theta/2} \end{array} \right) \\
   R_z(\theta)  & = & \left( \begin{array}{cc} e^{-i\theta/2} & 0 \\ 0  & e^{+i\theta/2} \end{array} \right) 
\label{eq:grover_phase_decoherence}
\end{eqnarray}
%
togehter with the so-called $i\mathrm{SWAP}$ two-qubit operator, which has the representation
%
\begin{equation}
i\mathrm{SWAP} = \left( \begin{array}{cccc} 1 & 0 & 0 & 0 \\ 0 & 0 & -i & 0 \\ 0 & -i & 0 & 0 \\ 0 & 0 & 0 & 1  \end{array}  \right)
\end{equation}
%
in the basis $(\ket{00},\ket{01},\ket{10},\ket{11})$. Often we will represent these gates schematically as shown in fig. \ref{fig:quantum_gates}.

\section{Superconduting Quantum Circuits}

\begin{SCfigure}
	\includegraphics[width=6cm]{"./material/figures/introduction/circuit_elements"}
	\caption{The three main circuit elements that are used to construct superconducting quantum circuits. Shown are a)a Josephson junction, b) an inductor and c)a capacitor.}
	\label{fig:SuperconductingCircuitElements}
\end{SCfigure}

In this section we will discuss several types of superconducting circuits that are most relevant to this work. We will start our discussion by introducing a general framework for treating classical circuit components such as inductances and capacitors quantum-mechanically and introduce the Josephson junction, the key element used to realize superconducting qubits.

\subsection{The Josephson junction}

The core element used to construct quantum circuits is the so-called {\it Josephson junction}, being equivalent to the transitor in classical circuits in significance. A Josephson junction is based on a disovery of Brian Josephson, which published a now-classical paper on quantum tunneling between weakly coupled superconductors \citep{josephson_possible_1962}. He found, that such a {\it weak link} between two superconductors could support a supercurrent $I$ described by the simple formula
%
\begin{equation}
I = I_c\sin{\phi}
\end{equation}
%
where $\phi = \phi_2-\phi$ and $\phi_1$ and $\phi_2$ are the superconducting phases at each side of the link. This simple equation, together with the current-phase relation of a Josephson junction,
%
\begin{equation}
U = \frac{\hbar}{2e}\frac{\partial \phi}{\partial t}
\end{equation}
%
yields a system exhibiting a  wealth of interesting physical phenomena that are used today in various applications in physics. The energy associated with the Josephson junction is given as
%
\begin{equation}
E = E_J(1-\cos{\phi})
\end{equation}
%
where $E_J = I_c \Phi_0/2\pi$ is the so-called {\it Joesephson energy}. In addition to this Josephson energy, the junction usually has an energy associated to the capacitance formed by the two seperated electrodes of the junction and given as $E_c = Q^2/2C$. 

\smallskip

For currents $I\ll I_c$, a Josephson junction behaves approximatively like a nonlinear inductor with inductance 
%
\begin{equation}
L_J = \frac{\Phi_0}{2\pi I_c \cos{\phi}}
\end{equation}
%
where $\Phi_0 = h/2e \approx 2.05 \times 10^{-15}\; \mathrm{Wb}$ is the so-called {\it magnetic flux quantum}. Later we will show how to make use of the properties of the Josephson junction to construct a qubit with it.

\subsection{Transmission Lines}

\begin{SCfigure}
	\includegraphics[width=6cm]{"./material/figures/introduction/transmission_lines"}
	\caption{a) The circuit diagram of a grounded tranmission line. b) The circuit diagram of a grounded coaxial tranmission line.}
	\label{fig:SampleCircuit}
\end{SCfigure}

Another circuit element that we will encounter many times in this work is the so-called {\it transmission line}. In the most general way, a transmission line is a structure with a large extension in one direction which is capable of transmitting electromagnetic waves along itself. Two possible symbols by which one usually designates tranmission lines in a circuit schematic are shown in fig. \ref{fig:tline_schematic}. A detailed treatment of the physics of tranmission lines can be found e.g. \cite{pozar_microwave_2011}. For this introduction we will skip these basics and start directly with the equation that describes the propagation of an electromagnetic wave along the extended dimension $z$ of the tranmission line and which is given as
%
\begin{eqnarray}
V(z,t) & = & \exp{\left(i\omega t\right)}\cdot\left(V^+ \exp{\left(-i\gamma z\right)}+V^-\exp{\left(i\gamma z\right)}\right) \\
I(z,t) & = & \frac{1}{Z_0}\exp{\left(i\omega t\right)}\cdot\left(V^+ \exp{\left(-i\gamma z\right)}-V^-\exp{\left(i\gamma z\right)}\right)
\end{eqnarray}
%
Here, $\gamma = \alpha+i\beta = \sqrt{(R+i\omega L)(G+i\omega C)}$ is the so-called {\it propagation constant} which describes the dispersion and damping of electromagnetic waves along the transmission line and $\omega$ the circular frequency of the electromagnetic wave. The voltages $V^+$ and $V^-$ correspond to waves traveling in different directions along the waveguide.

\smallskip

If we regard a CPW of fixed length $l$, we can model the voltages and currents at its end by using the formula\citep{pozar_microwave_2011}
%
\begin{equation}
\left( \begin{array}{c} V_1 \\ I_1 \end{array}\right) = \left( 
		\begin{array}{cc}
			\cos{\gamma l} & iZ_r \cos{\gamma l} \\
			i Y_r \sin{\gamma l} & \cos{\gamma l}
		\end{array}
		\right) \cdot \left(
		\begin{array}{c}
			V_2 \\ I_2
		\end{array}
		\right) \label{eq:cpw_abcd_matrix}
\end{equation}
%

\subsection{Transmission Line Resonators}

One can easily create a resonator using a coplanar waveguide as described in the last section. As an example, we will discuss an open-ended $\lambda / 2$ CPW resonator that is used in our experiments as well and thus highly relevant to this work. To create a resonator out of a CPW line, we terminate the line by an open and and connect it to a drive line through an input capacitance $C$, as shown in fig. \ref{fig:cpw_resonator}. To calculate the voltages and currents going into the CPW resonator, we can use eq. (\ref{eq:cpw_abcd_matrix}). Since the far end of the resonator is open we demand that $I_2=$ and thus obtain for the voltage $V_1$ and current $I_2$ the relation
%
\begin{eqnarray}
V_1 & = & \cos{\gamma l} V_2 \\
I_1 & = & i Y_r \sin{\gamma l} V_2
\end{eqnarray}
%
Hence, the impedance of the resonator is given as $Z_{res} = V_1/I_1 = i Z_r \cot{\gamma l}$, where $Z_r$ is the impedance of the tranmission line of which the resonator is made of. We couple this resonator to an input line through a gate capacitance $C_g$, thus the impendance seen from the input of the circuit is given as
%
\begin{equation}
Z_{in} = i Z_r \cot{\gamma l}-\frac{i}{\omega C_g} \label{eq:cpw_impedance}
\end{equation}
%
It is straightforward to calculate the $S_{11}$ reflection coefficient of the resonator when coupling it to an input line with impedance $Z_0$ as
%
\begin{equation}
S_{11} = \frac{Z_{in}-Z_0}{Z_{in}+Z_0} = \frac{Z_r\cot{\gamma l}-/\omega C_g-Z_0}{Z_r\cot{\gamma l}-/\omega C_g + Z_0}
\end{equation}
%
Now, when measuring the reflection of an incoming signal with voltage $V^+$ at frequency $f=2\pi \omega$ and phase $\phi_0$, the phase of the relected signal $\phi_{ref}$ will be simply given as $\phi_{ref}=\mathrm{Arg}[V^-/V^+]-\phi_0 = \mathrm{Arg}[S_{11}]-\phi_0$. Fig \ref{fig:cpw_resonator_phase} shows this phase for an exemplatory resonator, plotted for $f=2\pi\omega$ in reduced units of $[l/v]$, with impendances $Z_r,Z_0=50\;\mathrm{\Omega}$, $\alpha=0$ and with the resonator coupled to the input line through a normalized capacitance $C_g=10^{-3}/\omega\;[\mathrm{Hz}\cdot \mathrm{F}]$.

\begin{SCfigure}
	\includegraphics[width=10cm]{"./material/mathematica/cpw_lambda_over_4_phase_and_z"}
	\caption{This is the side caption which is maybe already to small to support such long text..}
	\label{fig:LambdaOver4ResonatorResponse}
\end{SCfigure}

We can calculate the quality factor of such an open-ended CPW resonator by modeling it as a series-LC circuit. Indeed,  if we regard the input impedance of the resonator in the vicinity of $\omega_0$ such that $\Delta \omega = \omega -\omega_0$ with $\Delta \omega \ll \omega_0$ and $\beta l = \pi +\pi\Delta \omega /\omega_0$, we obtain an effective impedance
%
\begin{equation}
Z_{in} = \frac{Z_{r}}{\alpha l + i(\Delta \omega \pi / \omega_0)}
\end{equation}
%
We can identify the quantities in this equation with the input impedance of a parallel LCR-resonator, which at $\omega\approx \omega_0$ is approximatively given as
%
\begin{equation}
Z_{in} = \frac{R}{1+2i Q \Delta \omega / \omega_0}
\end{equation}
%
with $W=\omega_0 RC$. This yields an effective resistance, inductance and capacitance for the transmission line resonator of
%
\begin{eqnarray}
R_{r} & = & \frac{Z_r}{\alpha l} \\
L_{r} & = & \frac{1}{\omega_0^2 C} \\
C_{r} & = & \frac{\pi}{2\omega_0 Z_r}
\end{eqnarray}
%
When coupling this resonator to an input transmission line of impedance $Z_0$ through a gate capacitance $C_g$ as before, the quality factor of the coupled (or {\it loaded}) resonator will be given as \citep{goppl_coplanar_2008}
%
\begin{equation}
Q_L = \omega_0^* \frac{C+C^*}{1/R_{r}+1/R^*}
\end{equation}
%
where we have introduced an effective resistance, capacitance and resonance frequency given as
%
\begin{eqnarray}
R^* & = & \frac{1+\omega_0^2 C_g^2 Z_0^2}{\omega_0^2 C_g^2 Z_0} \\
C^* & = & \frac{C_g}{1+\omega_0^2 C_g^2 Z_0^2} \\
\omega_0^* & = & \frac{1}{\sqrt{L_r(C_r+C_g)}}
\end{eqnarray}
%
Thus, we can effectively tune the quality factor of the resonator by changing the gate capacitance by which we couple it to the input tranmission line. 

\subsection{Quantization of Electrical Circuits}

In this section we will outline a general method to treat arbitary electrical circuits as the ones discussed before within the framework of quantum-mechanics, hence {\it quantizing} them. This introduction on circuit quantization presented in this chapter is based on an article by \cite{devoret_quantum_1995}.

\begin{SCfigure}
	\includegraphics[width=6cm]{"./material/figures/introduction/sample_circuit"}
	\caption{An exemplatory superconducing circuit made of a Josephson junction, a capacitor and a voltage source. The circuit topology can be described by one node (plus ground) and one branch.}
	\label{fig:SampleCircuit}
\end{SCfigure}

Fig. \ref{fig:SampleCircuit} shows an exemplatory circuit made of a Josephson junction, a capacitor and a voltage source. A circuit as this one is fully characterized by the parameters of its elements and its toplogy. The latter can be described -- following the laws of Kirchhoff -- as a set of nodes connected by a number of branches. In classical circuit theory, each branch $i$ is described by a voltage $V_i$ and a current $I_{i}$ flowing through it. The Kirchhoff laws demand that the sum of the branch voltages $V_i$ along any closed path must be zero, i.e. $\sum\limits_{\oint} V_i = 0$. Equivalently one may demand that the sum of currents flowing in and out of each node must be zero. For the quantization of electrical circuits it is usually more convenient to replace voltages and currents with branch charges and fluxes that are defined as
%
\begin{eqnarray}
\Phi_i(t) & = & \int\limits_{-\infty}^t V_i(t') dt' \\
Q_i(t) & = & \int\limits_{-\infty}^t I_i(t') dt'
\end{eqnarray}
%
In analogy with the Kirchhoff laws for the sums of currents and voltages along a closed branch, we can formulate a Kirchhoff law for the charges $Q_i$ at each node of the circuit, given as
%
\begin{eqnarray}
\sum\limits_{i} Q_i & = & Q_0 \label{eq:kirchhoff_charge}
\end{eqnarray}
%
where $Q_0$ is constant . To quantize such a circuit made up of non-dissipative elements we can follow the method given in \cite{yurke_quantum_1984}, writing the Lagrangian of the circuit as 
%
\begin{equation}
\begin{mathcal}L\end{mathcal} = \sum\limits_i V_i - \sum\limits_i T_i
\end{equation}
%
where $V_i$ and $T_i$ are the potential and kinetic energies associated to each circuit element. \todo{Clarify why kintetic energy is mapped to capacitive energy and potential energy to inductive energy}. For a circuit composed entirely of capacitors and inductor, the Lagrangian is given as
%
\begin{equation}
\begin{mathcal}L\end{mathcal} = \frac{1}{2}\sum\limits_i \frac{Q_i^2}{C_i}-\frac{1}{2}\sum\limits_i L \left( \frac{dQ_i}{dt}\right)^2 \label{eq:circuit_lagrangian}
\end{equation}
%
If needed, resistors can be described within the Lagrangian formalism by modeling them as transmission lines with a characteristic impedance matching their resistance \citep{yurke_quantum_1984}. From the Lagrangian as given in eq. (\ref{eq:circuit_lagrangian}) we obtain then the equations of motion of the system by variation of the action
%
\begin{equation}
\frac{\partial}{\partial t}\left( \frac{\partial \begin{mathcal}L\end{mathcal}}{\partial(\partial_t Q_i)}\right)-\frac{\partial \begin{mathcal}L\end{mathcal}}{\partial Q_i} = 0
\end{equation}
%
By imposing the charge-conservation equations as given by eq. (\ref{eq:kirchhoff_charge}) we obtain then a complete description of the underlying circuit. From the variable $Q_i$ we obtain the canonically conjugate momentum $\Phi_i$ by solving the equation
%
\begin{equation}
\Phi_i = \frac{\partial \begin{mathcal}L\end{mathcal}}{\partial(\partial_t Q_i)}
\end{equation}
%
First Quantization of the circuit variables can then be done by imposing commutation relations between the set of canonical variables $Q_i$ and $\Phi_i$ such that
%
\begin{eqnarray}
\left[Q_i(t),Q_j(t')\right] & = & 0 \\
\left[\Phi_i(t),\Phi_j(t') \right] & = & 0 \\
\left[Q_i(t),\Phi_i(t')\right] & = & i\hbar\delta_{ij}\delta(t-t') \label{eq:quantization_commutation_relations}
\end{eqnarray}
%
Having obtained $\Phi_i$ and $Q_i$, it is also trivial to obtain the Hamiltonian $\begin{mathcal}H\end{mathcal}$ of the system by applying the transformation
%
\begin{equation}
\begin{mathcal}H\end{mathcal} = \sum\limits_j \Phi_i \dot{Q}_i - \begin{mathcal}L\end{mathcal}
\end{equation}
%
\subsection{The Cooper Pair Box}

\begin{SCfigure}
	\includegraphics[width=6cm]{"./material/figures/introduction/cooper_pair_box"}
	\caption{The circuit schematic of a Split Cooper Pair Box. The device consists of two Josephson junctions arranged in a loop, where the capacitance of the junctions is modeled as the extra capacitor indicated on the left. By putting the two junctions in series with a gate capacitance one creates an island one which charges can accumulate. The gate voltage can be controlled by an external voltage source.}
	\label{fig:SampleCircuit}
\end{SCfigure}

A Transmon qubit is essentially a Cooper pair box (CPB) operated in the phase regime, where $E_J \gg E_C$. The Hamiltionian of the CPB can be written as \citep{cottet_implementation_2002}

\begin{equation}
\hat{H} = 4 E_C \left( \hat{n} - n_g\right)^2-E_J \cos{\hat{\phi}}
\end{equation}

where $E_C = e^2 / C_\Sigma$ is the charging energy with $C_\Sigma = C_J+C_B+C_g$ the total gate capacitance of the system, $\hat{n}$ is the number of Cooper pairs transferred between the islands, $n_g$ the gate charge, $E_J$ the Josephson energy of the junction and $\hat{\phi}$ the quantum phase across the junction.

\smallskip

The corresponding wavefunction $\Psi_k(\theta) = \bracket{\theta,k}$ will then satisfy the Schrödinger equation
%
\begin{equation}
E_k \Psi_k(\theta) = E_C(\frac{1}{i}\frac{\partial}{\partial \theta}-n_g)^2 \Psi_k(\theta) - E_J \cos{\left(\theta\right)}\Psi_k(\theta) \label{eq:cpb_schroedinger_equation}
\end{equation}
%
Since the potential $E_J\cos{(\theta)}\Psi_k(\theta)$ is periodic, we may demand that the solution will also exhibit a periodicity of the form
%
\begin{equation}
\Psi_k(\theta) = \Psi_k(\theta+2\pi)
\end{equation}
%
As it turns out, eq. (\ref{eq:cpb_schroedinger_equation}) can be mapped easily to the so-called {\it Mathieu differential equation}, which is given as
%
\begin{equation}
\frac{d^2y}{dx^2}+\left[a-2q\cos{(2x)}\right]y = 0
\end{equation}
%
Applying {\it Floquet's theorem} to this equation it can be shown that there exist solutions of the form
%
\begin{equation}
F(a,q,x) = \exp{\left(i\mu x\right)}P(a,q,x)
\end{equation}
%
The most general solutions to eq. (\ref{eq:cpb_schroedinger_equation}) are given as \citep{cottet_implementation_2002}
%
\begin{equation}
\Psi_k(r,q,\theta) = \mcal{C}_1\exp{\left(i n_g \theta \right)}\mcal{M}_C\left(\frac{4E_k}{E_C},-\frac{2E_J}{E_C},\frac{\theta}{2}\right)+\mcal{C}_2\exp{\left(i n_g \theta \right)}\mcal{M}_S \left(\frac{4 E_k}{E_C},-\frac{2 E_J}{E_C},\frac{\theta}{2}\right)
\end{equation}
%
with 
%
\begin{equation}
E_k = \frac{E_C}{4}\mcal{M}_A \left(r,-\frac{2 E_J}{E_C} \right)
\end{equation}
%
Here, $\mcal{M}_C$, $\mcal{M}_S$ are the {\it Mathieu functions} and $\mcal{M}_A$ corresponds to the eigenvalue.
This Hamiltonian can be solved exactly in the phase basis with the solutions being given as\citep{koch_charge-insensitive_2007,cottet_implementation_2002}

\begin{figure}[ht!]
	\includegraphics[width=\textwidth]{"./material/mathematica/cooper_pair_box_energies"}
	\caption{Energies of the first four energy level of the Cooper pair box for different ratios $E_J/E_C$, plotted as a function of the gate charge $n_g$. As can be seen, for $E_J \ll E_C$, the charge-dispersion curve becomes almost completely flat.}
	\label{fig:CooperPairBoxEnergies}
\end{figure}

%
\begin{equation}
E_m(n_g) = E_C a_{2[n_g+k(m,n_g)]}(-E_J/E_C)
\end{equation}
%
Here, $a_\nu(q)$ denotes  Mathieu's characteristic value and $k(m,n_g)$ is a function that sorts the eigenvalues. We denote the energy differences between individual energy level by $E_{ij} = E_j - E_i$. We also define the absolute and relative anharmonicities of the first two energy levels as $\alpha{12} \equiv E_{12}-E_{01}$ and $\alpha_r \equiv \alpha / E_{01}$. In the limit $E_J \gg E_C$ these anharmonicities are well approximated by $\alpha \simeq -E_C$ and $\alpha_r \simeq -(8E_J / E_C)^{-1/2}$.

\section{The Transmon Qubit}

The Transmon qubit is a Cooper pair box operated in the regime where $E_J \gg E_C$. As shown above, in this regime the charge dispersion of the energy levels of the Cooper pair box becomes flat, thus rendering the transition frequency $E_{01}$ almost insensitive to the value of the gate charge $n_g$. This reduced sensitivity to charge noise is highly advantageous in experiments. However, when increasing the ratio $E_J/E_C$, we also reduce the anharmonicity $\alpha_r$ of the qubit, therefore limiting the speed of gate operations that can be realized with this system (driving errors related to weak anharmonicity will be discussed more thoroughly in the Experiments section of this thesis). Remarkably though, $\alpha_r$ decreases only geometrically with $E_J/E_C$, whereas the sensitivity of the qubit to charge noise decreases exponentially with the ratio of Josephson and charging energy. Fig \ref{fig:TransmonAlphaAndT2} shows the relaxation rate and relative anharmonicity of  a Cooper Pair box for different values of $E_J/E_C$. As can be seen,

\section{Circuit Quantum Electrodynamics}

\begin{SCfigure}
	\includegraphics[width=11cm]{"./material/figures/introduction/cqed/cqed"}
	\caption{A typical circuit QED setup consisting of a Transmon qubit capacitively coupled to a $\lambda/2$ resonator.}
	\label{fig:CQED}
\end{SCfigure}


For readout and noise protection, the Transmon qubit is usually coupled to a harmonic oscillator which is usually realized as a lumped-elements resonator or a coplanar waveguide resonator. In the limit where the resonator capacity $C_r \gg C_\Sigma$ we can write the effective Hamiltonian of the system as

\begin{equation}
\hat{H} = \hbar \sum\limits_j \omega_j \ket{j}\bra{j} + \hbar \omega_r \hat{a}^\dagger \hat{a} + \hbar \sum\limits_{i,j} g_{ij} \ket{i}\bra{j}(\hat{a}+\hat{a}^\dagger) \label{eq:cqed_hamiltonian}
\end{equation}
Here, $\omega_r = 1/\sqrt{L_r C_r}$ gives the resonator frequency and $\hat{a}$ ($\hat{a}^\dagger$) are annihilation (creation) operators acting on oscillator states. The voltage of the oscillator is given by $V_{rms}^0 = \sqrt{\hbar \omega_r / 2 C_r}$ and the parameter $\beta$ gives the ratio between the gate capacitance and total capacitance, $\beta = C_g/C_\Sigma$. The coupling energies $g_{ij}$ are given as
\begin{equation}
\hbar g_{ij} = 2\beta e V_{rms}^0 \bra{i}\hat{n}\ket{j} = \hbar g_{ji}^*
\end{equation}
When the coupling between the resonator and the Transmon is weak $g_{ij} \ll \omega_r,E_{01}/h$ we can ignore the terms in eq. (\ref{eq:cqed_hamiltonian}) that describe simultaneous excitation or deexcitation of the Transmon and the resonator and obtain a simpler Hamiltonian in the so-called {\it rotating wave approximation} given as
\begin{equation}
\hat{H} = \hbar \sum\limits_j \omega_j \ket{j}\bra{j}+\hbar \omega_r \hat{a}^\dagger \hat{a} + \left( \hbar \sum\limits_i g_{i,i+1} \ket{i}\bra{i+1}\hat{a}^\dagger +H.c.\right)
\end{equation}

\subsection{Dispersive Limit \& Qubit Readout}

When the qubit frequency is far detuned from the resonator frequency direct qubit-resonator transition get exponentially supressed and the only interaction left between the two system is a dispersive shift of the transition frequencies. In this limit, the effective Hamiltonian of the system can be written as\citep{blais_cavity_2004,koch_charge-insensitive_2007}
\begin{equation}
\hat{H}_{eff} = \frac{\hbar \omega'_{01}}{2}\hat{\sigma}_z+\hbar(\omega_r' +\chi \hat{\sigma}_z)\hat{a}^\dagger \hat{a}
\end{equation}
Here, the resonance frequencies of both the qubit and the resonator are shifted and given as $\omega_r' = \omega_r-\chi_{12}/2$ and $\omega_{01}' = \omega_{01}+\chi_{01}$. The dispersive shift $\chi$ itself is given as
\begin{eqnarray}
\chi & = & \chi_{01}-\chi_{12}/2 \\
\chi_{ij} & = & \frac{g_{ij}^2}{\omega_{ij}-\omega_r} = \frac{(2\beta e V_{rms}^0)^2}{\hbar^2 \Delta_i}|\bra{i}\hat{n}\ket{i+1}|^2
\end{eqnarray}
The fact that $\chi_{01}$ and $\chi_{12}$ contribute to the total dispersive shift can cause the overall dispersive shift to become negative and even diverge at some particular working points.

\section{The Josephson Bifurcation Amplifier}

\begin{SCfigure}
	\includegraphics[width=5cm]{"./material/figures/introduction/nonlinear resonator"}
	\caption{The circuit model of the JBA.}
	\label{fig:jba_schematic}
\end{SCfigure}


\begin{SCfigure}
	\includegraphics[width=10cm]{"./material/figures/introduction/jba"}
	\caption{The Joesphson Bifurcation Amplifier used in this work.}
	\label{fig:CQED}
\end{SCfigure}

\citep{palacios-laloy_superconducting_2010}

\begin{equation}
[L_e+L_J (i)]\ddot{q}+R_e \dot{q}+\frac{q}{C_e} = V_e \cos{\left(\omega_m t\right)}
\end{equation}

Expanding this to second order in $L_J$ leads to the expression

\begin{equation}
\left(L_e+L_J\left[1+\frac{\dot{q}^2}{2 I_0^2}\right]\right)\ddot{q}+R_e \dot{q}+\frac{q}{C_e} = V_e \cos{\left( \omega_m t\right)}
\end{equation}

Defining the total inductance $L_t = L_e+L_J$, the participation ratio $p=L_J/L_t$, the resonance frequency $\omega_r = 1/\sqrt{L_t C_e}$ and the quality factor $Q = \omega_r L_t / R_e$ we can rewrite this as

\begin{equation}
\ddot{q}+\frac{\omega_r}{Q}\dot{q}+\omega_r^2 q + \frac{p \dot{q}^2 \ddot{q}}{2 I_0} = \frac{V_e}{L_t}\cos{\left(\omega_m t \right)}
\end{equation}

We can rewrite this equation in dimensionless units introducing the reduced variables $Q = \pi Z_0 / 2 R_e$, $\omega_r = \sqrt{1/(L_e+\phi_0/I_0)C_e}$, $\beta = (V_e/\phi_0 \omega_m)^2(pQ/2\Omega)^3$, $\Delta_m = \omega_r-\omega_m$, $\tau = t\Delta_m$ and $u(t) = \sqrt{pQ/2\Omega}\cdot q(t)\omega_m/I_0$, obtaining
%
\begin{align}
\frac{\Delta_m}{\omega_m}\frac{d^2 u}{d\tau^2}+\left(\frac{1}{Q \omega_m}+2i\right)\frac{du}{d\tau} \notag \\
 + \left[ 2\left(\frac{\omega_r^2-\omega_m^2}{2\omega_m \Delta_m}\right)+\frac{1}{Q\Delta m}-2|u|^2\right]u & =  2\sqrt{\beta} \label{eq:jba_theory_1}
\end{align}
%
In the limit where $Q \gg 1$, $\Delta_m\omega_m \ll 1$ such that $\omega_m+\omega_r \approx 2\omega_m$ and $\ddot{u}\ll\omega_m \dot{u}$ we can simplify eq. (\ref{eq:jba_theory_1}) to obtain
%
\begin{equation}
\frac{du}{d\tau} = -\frac{u}{\Omega}-iu\left(|u|^2-1\right)-i\sqrt{\beta}
\end{equation}
%
where we have defined $\Omega = 2Q\Delta_m/\omega_r$. The stationary solutions of this equation have the form
%
\begin{equation}
\frac{|u|^2}{\Omega^2}+|u|^2\left(|u|^2-1\right)^2 = \beta(\Omega)
\end{equation}
%
This equation can have one or two solutions depending on the parameter $\beta(\Omega)$. The region where multiple solutions exist is usually called the {\it bistability region} and is limited by the the two parameter boundaries $\beta^\pm(\Omega)$ that are given as
%
\begin{equation}
\beta^\pm(\Omega) = \frac{2}{27}\left[1+\left(\frac{3}{\Omega}\right)^2\pm\left(1-\frac{3}{\Omega^2}\right)^{3/2}\right]
\end{equation}
%